Este documento presenta el diseño del proyecto del sistema para una clínica, con desarrollo en el modelo iterativo, en la iteración 4. El proyecto está pensado para una empresa con las siguientes características:
La empresa cuenta con una clinica con 12 consultorios fijos, medicos, enfermeras, cajeros y gerentes, los cuales seran los usuarios del sistema. Así mismo el sistema esta pensado para una clinica que tenga una farmacia, por lo que debe contar con un control de ventas , control de inventario y pago de consultas. El sistema cuenta con un módulo para agendar citas y llevar un control de expediente medico.

El proyecto lo realiza el equipo Black::Sheep() para la materia de Análisis y Diseño Orientado a Objetos (ADOO) en la Escuela Superior de Cómputo del Instituto Politécnico Nacional con fecha 7 de mayo del 2016


%--------------------------------------------------
\section{Propósito}
El propósito de este documento es que el equipo de desarrollo tenga una guía para hacer el sistema que da solución a la problemática que se presenta en la clínica y que se quiere resolver. Así mismo,  tiene de propósito que  el equipo de desarrollo tenga documentados todos los procesos que realizará el sistema.
%--------------------------------------------------
\section{Alcance}

Organización
Lo que se mostrará en cada capítulo es lo siguiente:
\begin{itemize}
\item Lineamiento de diseño : estándares, notaciones

\item Modelo estático: estructura del proyecto

\item Modelo dinámico: como se va a ejecutar el sistema mediante diagramas de secuencia

\item Modelo del dominio del problema: modelado de datos, modelo entidad-relación, diccionario de datos, modelo de acceso de datos, listado de queries

\end{itemize}

%--------------------------------------------------
\section{Definiciones, acrónimos y abreviaturas}

\begin{itemize}
\item \textbf{Paciente: }Es una persona a la que se le proporciona atención médica.
\item \textbf{Clínica: }Es el lugar que se dedica a la observación y tratamiento médico de pacientes, así como a la venta de medicamentos.
\item \textbf{Consultorio: }Habitación ambientada para la revisión de pacientes.
\item \textbf{Consulta: }Es el proceso en el que un paciente asiste a un consultorio a ser revisado por un médico.
\item \textbf{Cita: }Es una consulta que está agendada previamente en un horario y consultorio definido
\item \textbf{Expediente (clínico): }Documento que describe la situación y antecedentes médicos de un paciente.
\end{itemize}
%--------------------------------------------------
%\section{Referencias}
%\begin{thebibliography}{9}
%\bibitem{nom}
 % Norma Oficial Mexicana,
  %\emph{NORMA Oficial Mexicana NOM-024-SSA3-2010, Que establece los objetivos funcionales y funcionalidades que deberán observar los productos de Sistemas de Expediente Clínico Electrónico para garantizar la interoperabilidad, procesamiento, interpretación, confidencialidad, seguridad y uso de estándares y catálogos de la información de los registros electrónicos en salud.},
%  2010.

%\end{thebibliography}
%--------------------------------------------------
