%--------------------------------------------------
\section{Propósito}
El propósito de este documento es que el equipo de desarrollo tenga una guía para hacer el sistema que da solución a la problemática que se presenta en la clínica y que se quiere resolver. Así mismo,  tiene de propósito que el cliente tenga clara la idea de qué es lo que se desarrollará, el alcance del sistema y su comportamiento.

%--------------------------------------------------
\section{Alcance}

Este documento va dirigido hacia el profesor de la Escuela Superior de  Cómputo,  Ulises Vélez Saldaña, que tendrá el rol del cliente en el desarrollo del proyecto.

%--------------------------------------------------
\section{Definiciones, acrónimos y abreviaturas}

\begin{itemize}
\item \textbf{Paciente: }Es una persona a la que se le proporciona atención médica.
\item \textbf{Clínica: }Es el lugar que se dedica a la observación y tratamiento médico de pacientes, así como a la venta de medicamentos.
\item \textbf{Consultorio: }Habitación ambientada para la revisión de pacientes.
\item \textbf{Consulta: }Es el proceso en el que un paciente asiste a un consultorio a ser revisado por un médico.
\item \textbf{Cita: }Es una consulta que está agendada previamente en un horario y consultorio definido
\item \textbf{Expediente (clínico): }Documento que describe la situación y antecedentes médicos de un paciente.
\end{itemize}
%--------------------------------------------------
\section{Referencias}
\begin{thebibliography}{9}

\bibitem{nom}
  Norma Oficial Mexicana,
  \emph{NORMA Oficial Mexicana NOM-024-SSA3-2010, Que establece los objetivos funcionales y funcionalidades que deberán observar los productos de Sistemas de Expediente Clínico Electrónico para garantizar la interoperabilidad, procesamiento, interpretación, confidencialidad, seguridad y uso de estándares y catálogos de la información de los registros electrónicos en salud.},
  2010.

\end{thebibliography}

%--------------------------------------------------
\section{Contenido y organización}
El contenido del presente documento se divide en 6 secciones que se describen a continuación:

\begin{itemize}
\item \textbf{Análisis del problema: }Se describe el contexto y el panorama del proceso de negocio actual que hay dentro de la clínica describiendo y resaltando los problemas que repercuten en el proceso de la clínica.
\item \textbf{Propuesta de solución: }Se especifica la solución que se plantea para el problema analizado en el capítulo anterior. 
\item \textbf{Modelo de negocio: }Se define la forma en que el nuevo proceso especificado en la solución trabajará sobre la clínica.
\item \textbf{Modelo de despliegue: }Se definen las características que tendrá el sistema, así como sus componentes de software y hardware.
\item \textbf{Modelo de comportamiento: }Se define la manera en que se comportará el sistema mediante el uso de diagramas de casos de uso junto con su especificación de cada uno de ellos.
\item \textbf{Modelo de interacción: }Se muestra la manera en que el usuario y el sistema estarán interactuando mediante pantallas de uso del sistema.
\end{itemize}