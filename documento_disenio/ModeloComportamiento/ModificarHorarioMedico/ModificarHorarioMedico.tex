|\begin{UseCase}{CU5}{Modificar horario médico}{Cuando un paciente tenga la necesidad de ser atendido por uno de los médicos de la clínica y necesite agendar una consulta médica, se realizará una cita. El sistema guardará los datos de la cita.}
	\UCitem{Version}{0.2}
  \UCitem{Actor}{Gerente}
  \UCitem{Proposito}{    
    El médico pueda ser esté disponible para ser seleccionado para la asignación de citas en un horario distinto al que tenía anteriormente.   
  }
  \UCitem{Entradas}{
    \begin{itemize}
      \item ID del paciente: entero positivo.              
      \item Rango de horas de trabajo: Rango con una hora de inicio de trabajo y de fin con formato hh:mm:ss am/pm.      
      \item Hora de comida (opcional): Hora en la que el médico no está presente en el consultorio para ir a comer. En caso de ser seleccionado ésta debe encontrarse dentro del rango de trabajo del médico.      
    \end{itemize}
  }
  \UCitem{Origen}{
    \begin{itemize}
      \item ID del paciente: de la sesión.      
      \item Hora de comida: Seleccionable en la pantalla.      
      \end{itemize}
  }
  \UCitem{Salidas}{
    \begin{itemize}
      \item Mensaje {\em MSG5a-}: “Cambio de horario exitoso”.      
    \end{itemize}
  }
  \UCitem{Destino}{
    \begin{itemize}
      \item Mensaje: Pantalla.      
    \end{itemize}
  }
  \UCitem{Precondiciones}{
    \begin{itemize}        
        \item Que el médico esté registrado y habilitado para trabajar dentro de la clínica. 
        \item Que el médico tenga días de trabajo registrados.                                          
    \end{itemize}    
  }
  \UCitem{Postcondiciones}{
    \begin{itemize}
        \item El médico está disponible para dar citas en un horario distinto al que tenía asignado.                        
    \end{itemize}
  }
  \UCitem{Observaciones}{
    La selección del rango de horas se podrá realizar por media hora, es decir, no se podrá seleccionar una hora de inicio de 8:35 am.
  }
  \UCitem{Errores}{
    \begin{itemize}
        \item \textbf{ERR1} Si el actor selecciona una hora de comida para el médico que se encuentra fuera de su rango de trabajo seleccionado, muestra mensaje {\bf MSG5b-} de la UI... y continúa con el paso 9. 
    \end{itemize}  	
	}
  \UCitem{Tipo de ejecución}{Secundaria, viene de CU1 Iniciar Sesión}
	\UCitem{Volatilidad}{Baja}
	\UCitem{Madurez}{Media}
	\UCitem{Prioridad}{Alta}
  \UCitem{Estado}{En edición}
	\UCitem{Autor}{Saúl Uriel Trujillo García}
	\UCitem{Revisor}{Gómez Moncada Demis}
\end{UseCase}

\begin{UCtrayectoria}{Principal}  
  \UCpaso[\UCactor] solicita la modificación de horario de un médico haciendo clic en el botón \IUbutton{Modificar horario médico} de la pantalla \IUref{UI5}{Pantalla principal de Gerente}.
  \UCpaso obtiene el nombre de cada uno de los médicos registrados y habilitados de la clínica del repositorio de datos.
  \UCpaso muestra en pantalla un listado con los nombres de los médicos que obtuvo del repositorio de datos en la 
  \UCpaso[\UCactor] selecciona del listado al médico del que le desea modificar su horario de disponibilidad para las citas.
  \UCpaso obtiene los días de la semana y horario para cada día en los que el médico seleccionado está disponible para dar citas.
  \UCpaso muestra los días de la semana y horario para cada día en la pantalla.....  
  \UCpaso[\UCactor] se dirige al apartado del día que desea modificar el horario.
  \UCpaso[\UCactor] modifica el rango del horario de trabajo para ese día.\Trayref{TA1}\Trayref{TA2}
  \UCpaso muestra el rango de horario para el día de la semana seleccionado.
  \UCpaso[\UCactor] modifica el rango de horario.
  \UCpaso[\UCactor] oprime el botón \IUbutton{Guardar cambios} de la pantalla.
  \UCpaso obtiene los datos proporcionados para hacer la actualización en el repositorio de datos.
  \UCpaso muestra \IUref{UI6}{Pantalla modificar horario listado médicos} y muestra mensaje {\bf MSG5a-}.     
\end{UCtrayectoria}

\begin{UCtrayectoriaA}{TA1}{El actor desea agregar una hora de comida al médico}			
			\UCpaso[\UCactor] oprime el botón \IUbutton{Agregar hora de comida} de \IUref{UI7}{Pantalla horario de médicos}
      \UCpaso[\UCactor] selecciona la hora de comida sobre el rango de horas de trabajo del médico.
      \UCpaso verifica que la hora de comida se encuentre dentro del rango de horas de trabajo del médico.\textbf{[ERR1]}
      \UCpaso Continúa en el paso 11 del \UCref{CU5}.
\end{UCtrayectoriaA}

\begin{UCtrayectoriaA}{TA2}{El actor desea quitar una hora de comida al médico}			
			\UCpaso[\UCactor] oprime el botón \IUbutton{Quitar hora de comida} de \IUref{UI7}{Pantalla horario de médicos}            
      \UCpaso Continúa en el paso 11 del \UCref{CU5}.
\end{UCtrayectoriaA}