\begin{UseCase}{CU2}{Agendar Cita}{Cuando un paciente tenga la necesidad de ser atendido por uno de los médicos de la clínica y necesite agendar una consulta médica, se realizará una cita. El sistema guardará los datos de la cita.}
	\UCitem{Version}{0.2}
  \UCitem{Actor}{Paciente}
  \UCitem{Proposito}{
    Que el Paciente pueda agendar una cita en el día y horario que desee para realizar un chequeo general, obtener el diagnóstico de una enfermedad o un certificado médico, siempre y cuando el sistema no tenga registrada una cita en el mismo día y hora.
  }
  \UCitem{Entradas}{
    \begin{itemize}
      \item ID del paciente: entero positivo.
      \item Fecha de cita: fecha con el formato dd/mm/aaaa.
      \item Hora de cita: hora con el formato hh:mm:ss.
      \item Consultorio: entero positivo entre 0 y 12.
    \end{itemize}
  }
  \UCitem{Origen}{
    \begin{itemize}
      \item ID del paciente: de la sesión.
      \item Fecha de cita: desde un calendario ordenado por mes.
      \item Hora de cita: desde una lista que se despliega al seleccionar el día.
      \item Consultorio: desde una lista.
      \end{itemize}
  }
  \UCitem{Salidas}{
    \begin{itemize}
      \item Mensaje [[no. de mensaje]]: “Registro de cita exitoso”.
      \item Correo electrónico de confirmación.
    \end{itemize}
  }
  \UCitem{Destino}{
    \begin{itemize}
      \item Mensaje: Pantalla.
      \item Correo electrónico: Servidor de correo electrónico.
    \end{itemize}
  }
  \UCitem{Precondiciones}{
    Que no haya una cita registrada en la misma fecha y hora, en el consultorio seleccionado.
  }
  \UCitem{Postcondiciones}{
    \begin{itemize}
      \item Se registra en el sistema la cita.
      \item La cita registrada aparece en la lista de citas registradas.
    \end{itemize}
  }
  \UCitem{Observaciones}{
    Ninguna
  }
  \UCitem{Errores}{
  	El actor introduce una fecha, hora y consultorio en los cuales ya existe una cita previamente registrada.
	}
  \UCitem{Tipo de ejecución}{Secundaria, viene de CU1 iniciar sesión}
	\UCitem{Volatilidad}{}
	\UCitem{Madurez}{}
	\UCitem{Estado}{}
	\UCitem{Autor}{Adrián Galindo}
	\UCitem{Revisor}{Rubén Murga}
\end{UseCase}

\begin{UCtrayectoria}{Principal}
		\UCpaso[\UCactor] se comunica con el sistema escribiendo la URL del sistema en el navegador de su preferencia.
		\UCpaso muestra \IUref{UI1}{Iniciar sesión}.
		\UCpaso[\UCactor] proprociona los datos requeridos.
		\UCpaso obtiene los datos y verifica la existencia del dato email en el repositorio de datos. [ERR2].
		\UCpaso verifica que el dato password corresponde al registro del usuario. [ERR3].
		\UCpaso otorga acceso.
		\UCpaso muestra \IUref{UI2}{Home}.
		\UCpaso[\UCactor] usa el sistema
\end{UCtrayectoria}
%-------------------------------------- TERMINA descripción del caso de uso.