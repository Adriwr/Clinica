\begin{UseCase}{CU12}{Cancelar cita}{Cuando un paciente quiera cancelar una de sus citas previamente agendadas, el sistema actualizará las citas del paciente. }
	\UCitem{Version}{0.3}
  \UCitem{Actor}{Paciente}
  \UCitem{Proposito}{
    Que el actor pueda cancelar una de sus citas agendadas si ya no podrá asistir o simplemente se haya equivocado en la fecha.
  }
  \UCitem{Entradas}{
    \begin{itemize}
    		\item Id de la cita: cadena aleatoria de caracteres formada por MongoDB
    \end{itemize}
  }
  \UCitem{Origen}{
    \begin{itemize}
      \item Id de la cita: Variable oculta en la UI3 Consultar citas.
      \end{itemize}
  }
  \UCitem{Salidas}{
    \begin{itemize}
      \item {\bf MSG12a-} Mensaje de notificación ``La cita se ha eliminado".
      \item {\bf MSG12b-} Mensaje de notificación ``No se encontró la cita".
    \end{itemize}
  }
  \UCitem{Destino}{
    \begin{itemize}
      \item Mensajes de notifiación: Pantalla.
    \end{itemize}
  }
  \UCitem{Precondiciones}{
    Que el paciente haya agendado al menos una cita.
  }
  \UCitem{Postcondiciones}{
    \begin{itemize}
      \item Se elimina la cita del actor en el sistema.
    \end{itemize}
  }
  \UCitem{Observaciones}{
    Ninguna
  }
  \UCitem{Errores}{
  	ERR1 - No se encuentra la cita del paciente a ser eliminada, muestra el mensaje {\bf MSG12b-} “No se encontró la cita".
	}
  \UCitem{Tipo de ejecución}{Secundaria, viene de CU3 Consultar Citas}
	\UCitem{Volatilidad}{Media}
	\UCitem{Madurez}{Media}
	\UCitem{Prioridad}{Alta}
  \UCitem{Estado}{En revisión}
	\UCitem{Autor}{Adrián Eduardo Galindo García}
	\UCitem{Revisor}{Rubén Murga Dionicio}
\end{UseCase}

\begin{UCtrayectoria}{Principal}
  \UCpaso[\UCactor] solicita consultar sus citas haciendo clic en la opción del menú “Consultar citas”.
  \UCpaso obtiene el registro de las citas que ha agendado el paciente
  \UCpaso muestra las citas agendadas del paciente en la UI3 Consultar citas paciente.
  \UCpaso[\UCactor] da click sobre el botón \IUbutton{ x Cancelar cita} de la cita que desea cancelar 
  \UCpaso verifica la existencia de la cita mediante el id.[ERR1]
  \UCpaso elimina la cita del sistema.
  \UCpaso notifica el resultado de la operación mostrando el mensaje {\bf MSG12a-} “La cita se ha eliminado”.
\end{UCtrayectoria}

%-------------------------------------- TERMINA descripción del caso de uso.