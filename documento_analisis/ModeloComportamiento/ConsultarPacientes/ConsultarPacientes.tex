%-------------------------------------- COMIENZA descripción del caso de uso.

\begin{UseCase}{CU21}{Consultar Paciente}{El gerente visualizará la información de todos los pacientes registrados en el sistema cuando lo necesite. El sistema mostrará la información de los pacientes.}
	\UCitem{Versión}{0.1}
    \UCitem{Actor}{Gerente}
    \UCitem{Propósito}{El gerente pueda observar los datos de los pacientes registrados en el sistema.}
    \UCitem{Entradas}{
        \begin{itemize}
            \item Email del gerente: cadena de texto compuesta de 5 partes ordenadas: 
            \begin{itemize}
                \item cadena de caracteres.
                \item carácter `@'.
                \item cadena de caracteres que identifica al servidor que brinda el servicio de correo electrónico.
                \item carácter `.'.
				\item cadena de caracteres representan el dominio del servidor de correo (Por ejemplo com, mx, etc.).
            \end{itemize}
            \item Tipo de usuario: número entero que corresponde al tipo de usuario que está usando el sistema. 4 valores posibles:
            \begin{itemize}
            	\item 0 Paciente.
            	\item 1 Enfermera.
            	\item 2 Médico.
            	\item 3 Gerente.
            \end{itemize}
            Para el CU21 - Consultar paciente, el tipo de usuario debe tener como valor 3.
        \end{itemize} 
    }
    \UCitem{Origen}{
        \begin{itemize}
            \item Email del gerente, tipo de usuario: Variable de sesión de la aplicación.
        \end{itemize}
    }
    \UCitem{Salidas}{
        \begin{itemize}
        \item Lista de pacientes: Arreglo de estructuras de datos que contienen lo siguiente:
	        \begin{itemize}
	        	\item Identificador del paciente: número entero.
				\item Nombre del paciente: tres cadenas de caracteres:
				\begin{itemize}
					\item Nombre(s) del paciente.
					\item Apellido paterno.
					\item Apellido materno.
		        \end{itemize}
		        \item Sexo: caracter 'H' o 'M'. 'H' si es Hombre, 'M' si es mujer.
		        \item Fecha de nacimiento: cadena de caracteres con el formato ``DD/MM/AAAA'', donde DD, MM y AAAA son número enteros que representan el día, mes y año de la fecha, respectivamente.
		        \item Dirección: Cadena de caracteres.
		        \item Teléfono particular: Cadena de 8 o 10 dígitos.
		        \item Teléfono para emergencias: Cadena de 8 o 10 dígitos.
		    \end{itemize}   
		\end{itemize}
	}
    \UCitem{Destino}{
		\begin{itemize}
			\item Nombre del paciente, Sexo, Fecha de nacimiento, Dirección, Teléfono particular, Teléfono para emergencias: Pantalla UI21 Consultar Pacientes.
			\item Identificador del paciente: Variable oculta del lado de cliente.
		\end{itemize}
	}
	
   	\UCitem{Precondiciones}{
   	    \begin{itemize}
			\item Existe al menos un paciente registrado en el sistema.
   	    \end{itemize}
   	}
    \UCitem{Postcondiciones}{Ninguna}
    \UCitem{Observaciones}{Ninguna}
    \UCitem{Errores}{
    	\begin{itemize}
	    	\item ERR1 - La entrada ``tipo de usuario'' es diferente de 3:
	    	\begin{itemize}
	    		\item El sistema muestra la pantalla UI0 Home.
	    		\item El sistema muestra el mensaje MSG0X ``Usuario no autorizado para realizar esta acción.''
	    		\item Fin del CU21.
	    	\end{itemize}
	    	\item ERR2 - No hay pacientes registrados:
	    	\begin{itemize}
				\item El sistema muestra el MSG21a ``No hay pacientes registrados.''.
				\item Fin del CU21.
			\end{itemize}
		
		\end{itemize}
	}
	\UCitem{Tipo de ejecución}{Secundaria, viene de CU1 iniciar sesión}
	\UCitem{Prioridad}{Media}
	\UCitem{Volatilidad}{Media}
	\UCitem{Madurez}{Media}
	\UCitem{Estado}{En revisión}
	\UCitem{Autor}{Demis Gómez Moncada}
	\UCitem{Revisor}{}
	
\end{UseCase}

\begin{UCtrayectoria}{}
		\UCpaso[\UCactor] selecciona mostrar los pacientes registrados dando click en el botón ``Mostrar Pacientes''. [ERR1]
		\UCpaso muestra la pantalla UI21 - Mostrar citas del paciente con la lista de los pacientes registrados. [ERR2].
\end{UCtrayectoria}
%-------------------------------------- TERMINA descripción del caso de uso.