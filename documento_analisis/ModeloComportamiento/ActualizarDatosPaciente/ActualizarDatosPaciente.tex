\begin{UseCase}{CU16}{Modificar Datos}{Cuando un paciente desee o tenga que actualizar su información personal, el sistema guardará los nuevos datos del paciente.}
	\UCitem{Version}{0.3}
  \UCitem{Actor}{Paciente}
  \UCitem{Proposito}{
    Que el paciente pueda modificar sus datos para contar con la información de contacto más actualizada.
  }
  \UCitem{Entradas}{
    \begin{itemize}
    		\item Id del paciente: cadema aleatoria de caracteres formada por MongoDB
            \item Nombre completo del paciente:  cadena(s) de caracteres (Nombre(s)).
            \item Apellidos del paciente:  cadena(s) de caracteres ( apellido paterno, apellido materno).
			\item Fecha de nacimiento del paciente: Fecha DD/MM/AAAA
            \item* Sexo del paciente: 1 carácter . ‘H’ para hombre, ‘M’ para mujer.
            \item Dirección
            \begin{itemize}
                \item Calle: Cádena de caracteres.
                \item Número: Cádena de caracteres.
                \item Colonia: Cádena de caracteres.
                \item Estado: Cádena de caracteres.
                \item Teléfono particular: Cádena de 8 a 10 caracteres.
                \item Teléfono de emergencia: Cádena de 8 a 10 caracteres.
            \end{itemize}    
        \end{itemize} 
  }
  \UCitem{Origen}{
    \begin{itemize}
      \item De la sesión de la aplicación 
      \begin{itemize}
                \item Id del paciente
      \end{itemize}    
      \item Del teclado 
      \begin{itemize}
            \item Nombre completo del paciente.
            \item Apellidos del paciente.
            \item Dirección
            \begin{itemize}
                \item Calle
                \item Número.
                \item Colonia.
                \item Estado.
                \item Teléfono particular.
                \item Teléfono de emergencia.
            \end{itemize}
      \end{itemize}
      \item De un calendario
      \begin{itemize}
                \item Fecha de nacimiento del paciente.
      \end{itemize}    
      \item De una lista de opciones
      \begin{itemize}
                \item Sexo del paciente.
      \end{itemize}          
      \end{itemize}
  }
  \UCitem{Salidas}{
    \begin{itemize}
      \item Mensaje de notificación ``Sus datos han sido actualizados".
      \item Mensaje de notificación ``No se encontraron los datos del paciente".
      \item Mensaje de notificación  ``Debe proporcionar todos los datos solicitados".
    \end{itemize}
  }
  \UCitem{Destino}{
    \begin{itemize}
      \item Mensajes de notificación: Pantalla.
    \end{itemize}
  }
  \UCitem{Precondiciones}{
   Que el paciente tenga datos registrados en el sistema
  }
  \UCitem{Postcondiciones}{
    \begin{itemize}
      \item Se actualizan los datos del paciente.
    \end{itemize}
  }
  \UCitem{Observaciones}{
    Ninguna
  }
  \UCitem{Errores}{
  	No se encontró el registro del paciente en el sistema, muestra el mensaje {\bf MSG2-} “No se encontraron los datos del paciente".
	}
  \UCitem{Tipo de ejecución}{Secundaria, viene de CU15 Consultar datos}
	\UCitem{Volatilidad}{Alta}
	\UCitem{Madurez}{Alta}
	\UCitem{Prioridad}{Baja}
  \UCitem{Estado}{En revisión}
	\UCitem{Autor}{Adrián Eduardo Galindo García}
	\UCitem{Revisor}{Rubén Adolfo Murga Dionicio}
\end{UseCase}

\begin{UCtrayectoria}{Principal}
  \UCpaso[\UCactor] solicita ver sus datos haciendo click en la opción del menú “Mis datos”.
  \UCpaso obtiene los datos personales del paciente.
  \UCpaso muestra los datos del paciente en la UI15A Consultar datos.
  \UCpaso[\UCactor] indica que quiere actualizar sus datos hacienndo click en el botón \IUbutton{Actualizar datos}.
  \UCpaso cambia la vista y muestra los datos del paciente en la UI15B Consultar datos.
  \UCpaso[\UCactor] ingresa los nuevos datos solicitados en el formulario.
  \UCpaso[\UCactor] da click en el botón \IUbutton{Guardar}.
  \UCpaso comprueba los datos solicitados esten completos. [Trayectoria A].
  \UCpaso verifica que exista el registro del paciente, obteniendo sus datos mediante la sesión.
  \UCpaso actualiza los datos dal paciente.
  \UCpaso notifica el resultado de la operación, muestra el mensaje {\bf MSG1-} “Sus datos han sido actualizados”. 
\end{UCtrayectoria}

\begin{UCtrayectoriaA}{A}{Datos completos}
  \UCpaso notifica mostrando el mensaje {\bf MSG3-} “Debe proporcionar todos los datos solicitados”.
  \UCpaso continúa transacción desde el paso 6.
\end{UCtrayectoriaA}

%-------------------------------------- TERMINA descripción del caso de uso.