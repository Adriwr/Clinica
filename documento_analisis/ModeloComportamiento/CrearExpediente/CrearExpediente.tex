% \IUref{IUAdmPS}{Administrar Planta de Selección}
% \IUref{IUModPS}{Modificar Planta de Selección}
% \IUref{IUEliPS}{Eliminar Planta de Selección}

% 


% Copie este bloque por cada caso de uso:
%-------------------------------------- COMIENZA descripción del caso de uso.

%\begin{UseCase}[archivo de imágen]{UCX}{Nombre del Caso de uso}{
\begin{UseCase}{Cu6}{Consultar expediente médico}{El actor desea consultar el expediente médico.}
	\UCitem{Versión}{0.1}
    \UCitem{Actor}{Médico,Paciente}
    \UCitem{Propósito}{Conocer los antecedentes médicos del segundo actor.}
    \UCitem{Entradas}{
    	\begin{itemize}
    		\item Email del paciente: cadena de texto compuesta de 4 partes [requerido]:  
    		\begin{itemize}
    			\item cadena de texto.
    			\item carácter ‘@’
    			\item cadena que identifica al servidor que brinda el servicio de correo electrónico
    			\item carácter ‘.’ y dominio
    		\end{itemize}
			\item Lista de datos personales [todos requeridos]
        \begin{itemize}
            \item Nombre: Constituido por nombre(s) y apellidos.Cádena de caracteres
            \item Sexo  Caracter 'H' o 'M' (hombre,mujer)
            \item Fecha de nacimiento: Formato de fecha DD/MM/AAAA
            \item Dirección
            \begin{itemize}
                \item Calle: Cádena de caracteres.
                \item Número: Cádena de caracteres.
                \item Colonia: Cádena de caracteres.
                \item Estado: Cádena de caracteres.
                \item País : Cádena de caracteres.
                \item Teléfono particular: Cádena de 8 a 10 caracteres.
                \item Teléfono de emergencia: Cádena de 8 a 10 caracteres.
            \end{itemize}    
		
		\end{itemize}
         
       \item Lista de enfermedades tenidas
        \begin{itemize}
            \item Código:código segun la cie10
            \item Nombre de enfermedad: Cadena de caracteres. 
            \item fecha diagnosticada: Formato DD/MM/AAAA
            \item Observaciones de la enfermedad: Cadena extensa de caracteres.
            \item Tratada: booleano 0 para 'No', 1 para 'Sí'.

        \end{itemize}
        
        \item Lista de cirugias tenidas
        \begin{itemize}
            \item Nombre de cirugia: Cadena de caracteres
            \item Año de la cirugia : Año 
        \end{itemize}
        
        \item Lista de alergias
        \begin{itemize}
            \item Nombre de sustancia a la que es alergico:Cadena de caracteres.
            \item Tratada: booleano 0 para 'No', 1 para 'Sí'.
        \end{itemize}
        
  \item Lista de antecedentes familiares
        
        \begin{itemize}
            \item nombre de familiar: Cadena extensa de caracteres consituido por nombre(s) y apellidos.
            \item Sexo del familiar: Caracter 'H' o 'M' (hombre,mujer)
            \item Edad de familiar: Número entero positivo.
            \item Nombre de enfermedad data a un familiar:Cadena de caracteres.
            \item Parentesco del familiar:Cadena de caracteres.
            \item Tratada: booleano 0 para 'No', 1 para 'Sí'.

        \end{itemize}
       \end{itemize}
    }
    \UCitem{Entradas}{
    	\begin{itemize}
        	 \item Lista de antecedentes personales [todos requeridos]
        
        \begin{itemize}
            \item Frecuencia en la que toma baño: Número de días.
            \item Frecuencia de cambio de ropa: Número de días.
            \item Personas viviendo en la misma casa: Número entero positivo.
            \item Servicios básicos: Lista de cadenas. 
            \item Alimentación: Descripción de tipo de alimentación. Cádena extensa de caracteres
            \item Fuma: Booleano
            \item Alcohol: Booleano
            \item Drogas: Booleano
            
        \end{itemize}
        
        \item Embarazos 
        
        \begin{itemize}
            \item Descripción de embarazo:Cadena de caracteres.
            \item Año del embarazo: Año
        \end{itemize}
        
        \item Lista de métodos anticonceptivos usados
        
        \begin{itemize}
             \item Anticonceptivo:Cadena de caracteres.
            \item Notas sobre el anticonceptivo:Cadena extensa caracteres.
        \end{itemize}
            
        \item Lista de mastografías
        
        \begin{itemize}
            \item Mastografía:Cadena de caracteres.
            \item Notas sobre la mastografía:Cadena extensa de caracteres.
        \end{itemize}
            
        \item Lista de papanicolaus
        
        \begin{itemize}
            \item Año de papanicolaou:Año
            \item Notas sobre el papanicolau:Cadena extensa de caracteres.
        \end{itemize}
        
		\end{itemize}
	}
    \UCitem{Origen}{
    	\begin{itemize}
    		\item Desde sesión: Correo electrónico de paciente.
    		\item Desde un catálogo: Código y nombre de enfermedad.
    		\item Desde teclado: todos los demás campos.
    	\end{itemize}
    }
    \UCitem{Salidas}{
    	\begin{itemize}
    		\item Mensaje de confirmación ``Expediente creado correctamente"
    	\end{itemize}
    }
    
    \UCitem{Destino}{Pantalla}
   	\UCitem{Precondiciones}{
   	    \begin{itemize}
	   	    \item Que el paciente no cuente con un expediente.
	   	    \item Que el médico esté atendiendo al paciente
   	    \end{itemize}
   	}
    \UCitem{Postcondiciones}{Se crea un expediente para el médico}
    \UCitem{Observaciones}{Ninguna}
    \UCitem{Errores}{
    	\begin{itemize}
    		\item ERR1: Si no se completaron todos los datos requeridos, se informa al actor mostrando el mensaje ``Complete todos los campos" y se continúa desde el paso 2
    		\item ERR2: Si los datos requeridos no tienen el formato correcto, se informa al actor mostrando el mensaje ``Formato de campos inválido" y se continúa desde el paso 2 
    	\end{itemize}
    }
	\UCitem{Tipo de ejecución}{Secundaria, viene de CU1 iniciar sesión}
	\UCitem{Prioridad}{Alta}
	\UCitem{Volatilidad}{Media}
	\UCitem{Madurez}{Baja}
	\UCitem{Estado}{Terminado}
	\UCitem{Autor}{David Alberto Pacheco Soto}
	\UCitem{Revisor}{Saúl Uriel Trujillo Gracía}
	
\end{UseCase}

\begin{UCtrayectoria}{Médico}
		\UCpaso[\UCactor] selecciona "Próximas citas" dando click en dicho apartado.
		\UCpaso Muestra un listado de todas las citas próximas que tiene agendadas. 
		\UCpaso[\UCactor] Da click en el nombre del paciente que es uno de los elementos en la lista.[RN13][RN8]
		\UCpaso Verifica que el paciente no tenga expediente.[TA1] [RN5]
		\UCpaso Solicita los datos listados en entradas mostrando la IU
		\UCpaso [\UCactor] Proporciona los datos
		\UCpaso [\UCactor] Confirma los datos haciendo click en el botón "registrar paciente"
		\UCpaso Verifica que todos los datos requeridos hayan sido completados
		\UCpaso Verifica que los datos proporcionados tengan el formato requerido
		\UCpaso Almacena los datos del expediente 
		\UCpaso Informa el resultado de la transacción
\end{UCtrayectoria}

\begin{UCtrayectoriaA}{TA1}{El paciente tiene expediente}
	\UCpaso Se ejecuta el caso de uso  "Consultar expediente"
	
\end{UCtrayectoriaA}

%-------------------------------------- TERMINA descripción del caso de uso.
