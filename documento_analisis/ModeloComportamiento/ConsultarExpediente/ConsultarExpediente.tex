% \IUref{IUAdmPS}{Administrar Planta de Selección}
% \IUref{IUModPS}{Modificar Planta de Selección}
% \IUref{IUEliPS}{Eliminar Planta de Selección}

% 


% Copie este bloque por cada caso de uso:
%-------------------------------------- COMIENZA descripción del caso de uso.

%\begin{UseCase}[archivo de imágen]{UCX}{Nombre del Caso de uso}{
\begin{UseCase}{Cu10}{Consultar expediente médico}{El actor desea consultar el expediente médico.}
    \UCitem{Versión}{0.1}
    \UCitem{Actor}{Médico,Paciente}
    \UCitem{Propósito}{Conocer los antecedentes médicos del segundo actor.}
    \UCitem{Entradas}{
        \begin{itemize}
            \item Email del paciente: cadena de texto compuesta de 4 partes:  
            \begin{itemize}
                \item cadena de texto.
                \item carácter ‘@’
                \item cadena que identifica al servidor que brinda el servicio de correo electrónico
                \item carácter ‘.’ y dominio
            \end{itemize}
        \end{itemize} 
    }
    \UCitem{Origen}{
        \begin{itemize}
            \item Desde actor paciente: Variable de sesión.
            \item Desde actor Médico: Repositorio de datos.
        \end{itemize}
    }
    \UCitem{Salidas}{
        \begin{itemize}
            \item Lista de datos personales
        \begin{itemize}
            \item Nombre: Constituido por nombre(s) y apellidos.Cádena de caracteres
            \item Sexo  Caracter 'H' o 'M' (hombre,mujer)
            \item Fecha de nacimiento: Formato de fecha DD/MM/AAAA
            \item Dirección
            \begin{itemize}
                \item Calle: Cádena de caracteres.
                \item Número: Cádena de caracteres.
                \item Colonia: Cádena de caracteres.
                \item Estado: Cádena de caracteres.
                \item País : Cádena de caracteres.
                \item Teléfono particular: Cádena de 8 a 10 caracteres.
                \item Teléfono de emergencia: Cádena de 8 a 10 caracteres.
            \end{itemize}    
        
        \end{itemize}
        
       \item Lista de enfermedades tenidas
        \begin{itemize}
            \item Código:código segun la cie10
            \item Nombre de enfermedad: Cadena de caracteres. 
            \item fecha diagnosticada: Formato DD/MM/AAAA
            \item Observaciones de la enfermedad: Cadena extensa de caracteres.
            \item Tratada: booleano 0 para 'No', 1 para 'Sí'.

        \end{itemize}
        
        \item Lista de cirugias tenidas
        \begin{itemize}
            \item Nombre de cirugia: Cadena de caracteres
            \item Año de la cirugia : Año 
        \end{itemize}
        
        \item Lista de alergias
        \begin{itemize}
            \item Nombre de sustancia a la que es alergico:Cadena de caracteres.
            \item Tratada: booleano 0 para 'No', 1 para 'Sí'.
        \end{itemize}
        
  \item Lista de antecedentes familiares
        
        \begin{itemize}
            \item nombre de familiar: Cadena extensa de caracteres consituido por nombre(s) y apellidos.
            \item Sexo del familiar: Caracter 'H' o 'M' (hombre,mujer)
            \item Edad de familiar: Número entero positivo.
            \item Nombre de enfermedad data a un familiar:Cadena de caracteres.
            \item Parentesco del familiar:Cadena de caracteres.
            \item Tratada: booleano 0 para 'No', 1 para 'Sí'.

        \end{itemize}
       \end{itemize}
    }
    \UCitem{Salidas}{
        \begin{itemize}
             \item Lista de antecedentes personales
        
        \begin{itemize}
            \item Frecuencia en la que toma baño: Número de días.
            \item Frecuencia de cambio de ropa: Número de días.
            \item Personas viviendo en la misma casa: Número entero positivo.
            \item Servicios básicos: Lista de cadenas. 
            \item Alimentación: Descripción de tipo de alimentación. Cádena extensa de caracteres
            \item Fuma: Booleano
            \item Alcohol: Booleano
            \item Drogas: Booleano
            
        \end{itemize}
        
        \item Embarazos 
        
        \begin{itemize}
            \item Descripción de embarazo:Cadena de caracteres.
            \item Año del embarazo: Año
        \end{itemize}
        
        \item Lista de métodos anticonceptivos usados
        
        \begin{itemize}
             \item Anticonceptivo:Cadena de caracteres.
            \item Notas sobre el anticonceptivo:Cadena extensa caracteres.
        \end{itemize}
            
        \item Lista de mastografías
        
        \begin{itemize}
            \item Mastografía:Cadena de caracteres.
            \item Notas sobre la mastografía:Cadena extensa de caracteres.
        \end{itemize}
            
        \item Lista de papanicolaus
        
        \begin{itemize}
            \item Año de papanicolaou:Año
            \item Notas sobre el papanicolau:Cadena extensa de caracteres.
        \end{itemize}
        
        \end{itemize}
}
    \UCitem{Salida}{
        \begin{itemize}
            \item Lista de consultas
        
        \begin{itemize}
            \item Fecha de consulta:DD/MM/AAAA
            \item Sintomas:Cadena extensa de caracteres.
            \item Peso: Médido en KG.
            \item Talla: Médida en talla de México
            \item Temperatura: Médida en grados centigrados
            \item Frecuencia cardiaca:Médido en número entero por minúto.
            \item Presión mínima:"mm de Hg" (milímetros de mercurio)
            \item Presión máxima:"mm de Hg" (milímetros de mercurio) 
            \item Diagnostico:Cadena extensa de caracteres.
            \item Observaciones del diagnostico:Cadena extensa de caracteres.
            \item Recomendación:Cadena extensa de caracteres.
            \item Duración de tratamiento: Medido en días.
            \item Nombre comercial del medicamento recetado:Cadena de caracteres.
            \item Frecuencia de uso: número de horas entre uso 
            \item Duración de uso: Días.
        \end{itemize}
        \end{itemize}
    }
    \UCitem{Destino}{Pantalla}
    \UCitem{Precondiciones}{
        \begin{itemize}
            \item Para actor paciente:
            \begin{itemize}
                \item Haber tenido su primera consulta.
            \end{itemize}
            \item para actor médico:
            \begin{itemize}
                \item Que el paciente cuente con un expediente.
            \end{itemize}
        \end{itemize}
    }
    \UCitem{Postcondiciones}{Ninguna}
    \UCitem{Observaciones}{
        Si el paciente no cuenta con un expediente el médico debe crearle uno.
    }
    \UCitem{Errores}{
    ERR1:
    }
    \UCitem{Tipo de ejecución}{Secundaria, viene de CU1 iniciar sesión}
    \UCitem{Prioridad}{Alta}
    \UCitem{Volatilidad}{Baja}
    \UCitem{Madurez}{Baja}
    \UCitem{Estado}{En edición}
    \UCitem{Autor}{Rubén Murga Dionicio}
    \UCitem{Revisor}{David Pacheco Soto}
    
\end{UseCase}

\begin{UCtrayectoria}{Médico}
        \UCpaso[\UCactor] selecciona "Próximas citas" dando click en dicho apartado.
        
        \UCpaso Muestra un listado de todas las citas próximas que tiene agendadas. 
        \UCpaso[\UCactor] Da click en el nombre del paciente que es uno de los elementos en la lista.[RN13][RN8]
        \UCpaso Muestra todos los datos de salida especificados en el apartado de salias.[TA1]
\end{UCtrayectoria}

\begin{UCtrayectoriaA}{TA1}{Datos sin valor}
    \UCpaso El sistema muestra los datos que no tienen valor como ``Sin valor''
    
\end{UCtrayectoriaA}

\begin{UCtrayectoria}{Paciente}
        \UCpaso[\UCactor] Selecciona ``Mi expediente'' dando click en dicho apartado.[TA2][RN8]
        \UCpaso Muestra todos los datos de salida especificados en el apartado de salias.[ERR1]
\end{UCtrayectoria}
%-------------------------------------- TERMINA descripción del caso de uso.
