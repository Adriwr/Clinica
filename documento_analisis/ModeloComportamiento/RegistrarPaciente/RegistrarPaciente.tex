% \IUref{IUAdmPS}{Administrar Planta de Selección}
% \IUref{IUModPS}{Modificar Planta de Selección}
% \IUref{IUEliPS}{Eliminar Planta de Selección}

% 


% Copie este bloque por cada caso de uso:
%-------------------------------------- COMIENZA descripción del caso de uso.

%\begin{UseCase}[archivo de imágen]{UCX}{Nombre del Caso de uso}{
\begin{UseCase}{CU4}{Registrar paciente
}{Cuando un paciente requiera una consulta por primera vez se registrarán sus datos personales en el sistema. El sistema almacenará los datos personales del paciente.}
	\UCitem{Versión}{0.1}
    \UCitem{Actor}{Enfermera,Paciente}
    \UCitem{Propósito}{El paciente podrá agendar citas y consultar su expediente.}
    \UCitem{Entradas}{
        \begin{itemize}
            \item *Email del paciente: cadena de texto compuesta de 4 partes:  
            \begin{itemize}
                \item cadena de texto.
                \item carácter ‘@’
                \item cadena que identifica al servidor que brinda el servicio de correo electrónico
                \item carácter ‘.’ y dominio
            \end{itemize}
            \item* Nombre completo del paciente:  3 cadenas de caracteres (Nombre(s), apellido paterno, apellido materno).
			\item *Fecha de nacimiento del paciente: Fecha DD/MM/AAAA
            \item *Sexo del paciente: 1 carácter . ‘H’ para hombre, ‘M’ para mujer.
            \item Dirección
            \begin{itemize}
                \item *Calle: Cádena de caracteres.
                \item Número: Cádena de caracteres.
                \item *Colonia: Cádena de caracteres.
                \item *Estado: Cádena de caracteres.
                \item *País : Cádena de caracteres.
                \item Teléfono particular: Cádena de 8 a 10 caracteres.
                \item Teléfono de emergencia: Cádena de 8 a 10 caracteres.
            \end{itemize}    
            \item *Contraseña:  Cadena de longitud mínima de 8 caracteres
        \end{itemize} 
    }
    \UCitem{Origen}{
        \begin{itemize}
            \item Teclado: Email, Nombre, Calle, Número, Colonia, Teléfono particular, Teléfono de emergencia, Contraseña
            \item Select(tipo de input html) :Sexo,País, Estado.
        \end{itemize}
    }
    \UCitem{Salidas}{Mensaje:``Paciente registrado correctamente''}
    \UCitem{Destino}{Pantalla}
   	\UCitem{Precondiciones}{
    No debe existir un paciente registrado con el correo electrónico ingresado.}
    \UCitem{Postcondiciones}{
    	\begin{itemize}
			\item El paciente queda registrado en el repositorio de datos.
		\end{itemize}
    }
    \UCitem{Observaciones}{
    \begin{itemize}
    \item La Pantalla para registrar paciente nuevo de la enfermera y del paciente es la misma.
    \item Los elementos marcados con '*' son obligatorios de llenar.
    \end{itemize}
    	
    }
    \UCitem{Errores}{}
	\UCitem{Tipo de ejecución}{
    \begin{itemize}
		\item Para la enfermara: Secundaria, viene de CU1 iniciar sesión
        \item Para el paciente: Primaria
	\end{itemize}
    }
	\UCitem{Prioridad}{Alta}
	\UCitem{Volatilidad}{Intermedia}
	\UCitem{Madurez}{Baja}
	\UCitem{Estado}{Terminado}
	\UCitem{Autor}{Rubén Murga Dionicio}
	\UCitem{Revisor}{David Pacheco Soto}
	
\end{UseCase}

\begin{UCtrayectoria}{Enfermera}
		\UCpaso[\UCactor] selecciona "Registrar  paciente" dando click en dicho apartado.
		\UCpaso solicita al actor los datos del paciente mostrando un formulario mostrando la UI ``UIRegPaciente" monstrada en el apartado 7.0.7
        \UCpaso [\UCactor] Proporciona los datos al sistema ingresandolos en el formulario.
         \UCpaso [\UCactor] Confirma la operación haciendo click en el botón ``Guardar"
         \UCpaso Verifica que el nombre sea una cadena de caracteres [TAValida]
         \UCpaso Verifica que la fecha de nacimiento sea una fecha válida [TAValida]
         \UCpaso Verifica que el teléfono sea numérico y de longitud de 8 o más caracteres [TAValida]
         \UCpaso El sistema verifica que el correo electrónico sea válido [TAValida]
         \UCpaso Verifica que el correo electrónico no pertenezca a un paciente ya registrado [TACorreo].
		\UCpaso Verifica que la contraseña coincida con el otro campo de contraseña [TAValida]
        \UCpaso Muestra mensaje.
        

         
         



        

        
        
\end{UCtrayectoria}

\begin{UCtrayectoriaA}{TAValida}{Un campo del formulario contiene datos incorrectos}
\UCpaso notifica el campo incorrecto al usuario mostrando un mensaje en pantalla.
\UCpaso notifica el campo incorrecto al usuario mostrando un mensaje en pantalla.
\UCpaso El sistema muestra el formulario.
\UCpaso Continúa transacción desde paso 5.
	
	
\end{UCtrayectoriaA}


\begin{UCtrayectoriaA}{TACorreo}{Ya existe un paciente registrado con el correo electrónico ingresado.}
\UCpaso Notifica el campo de correo electrónico mostrando el mensaje "Paciente ya registrado" en pantalla.
\UCpaso Solicita ingresar un correo electrónico diferente en el formulario.

\UCpaso [\UCactor] El actor ingresa otro correo electrónico en el campo “Correo electrónico” del formulario.

\UCpaso Continúa transacción desde paso 9.
\end{UCtrayectoriaA}

%-------------------------------------- TERMINA descripción del caso de uso.