% \IUref{IUAdmPS}{Administrar Planta de Selección}
% \IUref{IUModPS}{Modificar Planta de Selección}
% \IUref{IUEliPS}{Eliminar Planta de Selección}

% 


% Copie este bloque por cada caso de uso:
%-------------------------------------- COMIENZA descripción del caso de uso.

%\begin{UseCase}[archivo de imágen]{UCX}{Nombre del Caso de uso}{
\begin{UseCase}{CU1}{Iniciar sesión}{
		Cuando el actor tenga la necesidad de utilizar el sistema, este último verificará que esté autorizado para su uso. El sistema habilitará las opciones disponibles dependiendo del usuario.}
	\UCitem{Versión}{0.1}
    \UCitem{Actor}{Todo usuario del sistema}
    \UCitem{Propósito}{Acceder a las funcionalidades del sistema.}
    \UCitem{Entradas}{
      \begin{itemize}
      \item E-mail : cadena de texto compuesta de 4 partes : 
      \begin{itemize}
          \item cadena de texto que el actor proporciona.
          \item carácter ‘@’
          \item cadena que identifica al servidor que brinda el servicio de correo electrónico
      \item carácter ‘.’ y dominio
      \end{itemize}
      \item Password : cadena de texto brindada por el actor.
      \end{itemize}
    }
    \UCitem{Origen}{Teclado: E-mail, password.}
    \UCitem{Salidas}{Pantalla principal.}
    \UCitem{Destino}{Pantalla.}
   	\UCitem{Precondiciones}{ Estar registrado en el sistema.  }
    \UCitem{Postcondiciones}{El actor se encuentra identificado en el sistema. }
    \UCitem{Observaciones}{El sistema debe mostrar diferentes opciones dependiendo del tipo de usuario que inicio sesión.
    \begin{itemize}
		\item Opciones de menú para tipo de usuario Paciente:
			\begin{itemize}
				\item Agendar nueva cita
                \item Consultar expediente
				\item Consultar citas
			\end{itemize}
       	\item Opciones de menú para tipo de usuario Enfermera:
			\begin{itemize}
				\item Consultar citas
                \item Consultar médico
				\item Registrar paciente
                \item Agendar nueva cita
			\end{itemize}
        \item Opciones de menú para tipo de usuario Gerente:
			\begin{itemize}
				\item Modificar horarios de médicos
                \item Consultar actividad en la clínica
			\end{itemize}
       \item Opciones de menú para tipo de usuario Médico:
			\begin{itemize}
				\item Crear expediente nuevo
                \item Modificar expediente del paciente
				\item Consultar expediente del paciente
			\end{itemize}
       \item Opciones de menú para tipo de usuario Cajero:
			\begin{itemize}
                \item Registrar pago cita
				\item Registrar pago medicamentos
			\end{itemize}   
	\end{itemize}
    }
    \UCitem{Errores}{
	\begin{itemize}
		\item Err2:Si el dato “e-mail” proporcionado no se encuentra registrado en el repositorio de datos del sistema, se muestra mensaje “Usuario no registrado en el sistema” y continúa con paso 2. 
        \item Err3: Si la entrada “password” no corresponde con la entrada “e-mail” dentro del repositorio de datos, se muestra el mensaje “Usuario y/o contraseña incorrecto y continúa con paso 2.
	\end{itemize}
	}
\end{UseCase}

\begin{UCtrayectoria}{Principal}
		\UCpaso[\UCactor] se comunica con el sistema escribiendo la URL del sistema en el navegador de su preferencia.
		\UCpaso muestra \IUref{UI1}{Iniciar sesión}.
		\UCpaso[\UCactor] proprociona los datos requeridos.
		\UCpaso obtiene los datos y verifica la existencia del dato email en el repositorio de datos. [ERR2].
		\UCpaso verifica que el dato password corresponde al registro del usuario. [ERR3].
		\UCpaso otorga acceso.
		\UCpaso muestra \IUref{UI2}{Home}.
		\UCpaso[\UCactor] usa el sistema
\end{UCtrayectoria}
%-------------------------------------- TERMINA descripción del caso de uso.

%------------ empieza otro caso
\begin{UseCase}{CU2}{Agendar Cita
}{Cuando un paciente tenga la necesidad de ser atendido por uno de los médicos de la clínica y necesite agendar una consulta médica, se realizará una cita. El sistema guardará los datos de la cita.
}
	\UCitem{Versión}{0.2}
    \UCitem{Actor}{Paciente}
    \UCitem{Propósito}{Que el Paciente pueda agendar una cita en el día y horario que desee, siempre y cuando el sistema no tenga registrada otra cita.}
    \UCitem{Entradas}{
    \begin{itemize}
		\item ID del paciente: entero positivo. 
    	\item Fecha de cita: fecha (dd/mm/aaaa)
		\item Hora de cita: hora (hh:mm)
		\item Consultorio: entero positivo entre 0 y 12
	\end{itemize}
    }
    \UCitem{Origen}{
    	\begin{itemize}
			\item ID del paciente:Desde una lista
            \item Fecha de cita: Desde un calendario
            \item Hora de cita: Desde una lista
            \item Consultorio: Desde una lista
		\end{itemize}
    }
    \UCitem{Salidas}{
    	\begin{itemize}
			\item Mensaje [[no. de mensaje]]: “Registro de consulta exitoso”.
            \item Correo electrónico de confirmación.
		\end{itemize}
    }
    \UCitem{Destino}{
    \begin{itemize}
		\item Mensaje: Pantalla.
		\item Correo: Servidor de correo electrónico.
	\end{itemize}
    }
   	\UCitem{Precondiciones}{Que no haya una cita registrada en la misma fecha y hora, con el mismo doctor.}
    \UCitem{Postcondiciones}{ 
    	\begin{itemize}
			\item Se registra en el sistema la cita.
			\item No se puede agendar una cita en la misma fecha y hora, con el mismo doctor que la cita registrada.
			\item La cita registrada aparece en la lista de citas registradas.
		\end{itemize}
    }
    \UCitem{Observaciones}{ Ninguna
}
	    \UCitem{Errores}{El actor introduce una fecha, hora y doctor ya 			ocupados.
		}

	    \UCitem{Tipo de ejecución}{Secundaria, viene de CU1 iniciar sesión}

\end{UseCase}
\begin{UCtrayectoria}{Principal}
	\UCpaso[\UCactor] solicita agendar cita haciendo clic en el botón "Agendar cita".
	\UCpaso calcula el número de citas disponibles para cada día.
	\UCpaso informa la cantidad de citas disponibles por día mostrándolas en un calendario.
	\UCpaso[\UCactor] indica el día de la cita deseado haciendo click sobre el día.
	\UCpaso obtiene el consultorio, nombre del doctor y la hora para cada cita disponible.
	\UCpaso informa al usuario la información de las consultas disponibles para ese día mostrándola en una lista.
	\UCpaso[\UCactor] indica la cita deseada haciendo click sobre un elemento de la lista.
	\UCpaso solicita confirmar el registro de la cita mostrando la alerta "Confirmar cita".
	\UCpaso[\UCactor] confirma la cita haciendo click sobre el botón "Confirmar".
	\UCpaso guarda la cita registrándola en la base de datos.
	\UCpaso notifica el resultado de la operación mostrando la información de la cita.
\end{UCtrayectoria}
	%------------ empieza otro caso
	\begin{UseCase}{CU3}{Consultar citas}{El paciente desea saber la fecha, consultorio y médico de la cita que agendó previamente. El sistema mostrará la información de las citas agendadas.}
		\UCitem{Versión}{0.1}
	    \UCitem{Actor}{Paciente}
	    \UCitem{Propósito}{Conocer la fecha, consultorio y médico por el cual el actor será atendido.}
	    \UCitem{Entradas}{
	    -
	    }
	    \UCitem{Origen}{-}
	    \UCitem{Salidas}{
        	\begin{itemize}
				\item Fecha: cadena de caracteres que representan una fecha con formato DD/MM/AAAA.
				\item Número de consultorio: Número entero.
				\item Nombre del médico: Cadena de caracteres (Nombre(s), Apellido Paterno, Apellido Materno).

			\end{itemize}
        }
	    \UCitem{Destino}{Pantalla}
	   	\UCitem{Precondiciones}{Haber agendado una cita previamente.}
	    \UCitem{Postcondiciones}{Ninguna}
	    \UCitem{Observaciones}{Ninguna}
	    \UCitem{Errores}{
        \begin{itemize}
			\item ER1. 
            \item ER2. No hay citas registradas:
			\begin{itemize}
				\item El sistema muestra la pantalla “No hay citas registradas”.
				\item FIN.

			\end{itemize}
		\end{itemize}
        }
        	    \UCitem{Tipo de ejecución}{Secundaria, viene de CU1 iniciar sesión}

	\end{UseCase}
\begin{UCtrayectoria}{Principal}
	\UCpaso[\UCactor] selecciona consultar las citas dando click en el botón "Consultar mis citas".
	\UCpaso muestra una lista de todas las citas que tiene agendadas el actor a partir del día de hoy. [ERR2].
\end{UCtrayectoria}

	%------------ empieza otro caso
	\begin{UseCase}{CU4}{ Registrar paciente
}{Cuando un paciente requiera una consulta por primera vez se registrarán sus datos personales en el sistema. El sistema almacenará los datos personales del paciente.
}
		\UCitem{Versión}{0.1}
	    \UCitem{Actor}{Enfermera}
	    \UCitem{Propósito}{El paciente podrá agendar citas y consultar su expediente.}
	    \UCitem{Entradas}{
	    \begin{itemize}
			\item 
		    \item Nombre completo del paciente:  3 cadenas de caracteres (Nombre(s), apellido paterno, apellido materno).
	    	\item Fecha de nacimiento del paciente: Fecha
			\item Sexo del paciente: 1 carácter . ‘H’ para hombre, ‘M’ para mujer.
			\item Dirección del paciente: Cadena de caracteres.
			\item Teléfono del paciente: Cadena de dígitos de longitud mínima de 8 a 10 caracteres
			\item Correo electrónico del paciente: Correo electrónico válido.
			\item Contraseña del paciente: Cadena de longitud mínima de 8 caracteres

		\end{itemize}
	    }
	    \UCitem{Origen}{Teclado}
	    \UCitem{Salidas}{Mensaje “Paciente Registrado correctamente”.}
	    \UCitem{Destino}{Pantalla}
	   	\UCitem{Precondiciones}{No debe existir un paciente registrado con el correo electrónico ingresado.}
	    \UCitem{Postcondiciones}{ El paciente queda registrado en el sistema
}
	    \UCitem{Observaciones}{Ninguna}
	    \UCitem{Errores}{
        \begin{itemize}
			\item ERR1
		\end{itemize}
        }
        	    \UCitem{Tipo de ejecución}{Secundaria, viene de CU1 iniciar sesión}

	\end{UseCase}
\begin{UCtrayectoria}{Principal}
	\UCpaso[\UCactor] solicita registrar un nuevo paciente haciendo click sobre el botón ``Registrar paciente''.
	\UCpaso solicita al actor los datos del paciente mostrando la UI3 Registrar paciente.
	\UCpaso[\UCactor] proporciona los datos al sistema ingresándolos en el formulario.
	\UCpaso[\UCactor] confirma la operación haciendo click en el botón “Registrar paciente”.
	\UCpaso verifica que el nombre sea una cadena de caracteres con longitud mayor o igual a 10 caracteres. [Trayectoria A].
	\UCpaso verifica que la fecha de nacimiento sea una fecha válida. [Trayectoria A].
	\UCpaso verifica que  el sexo del paciente sea el caracter 'H' o ‘M’. [Trayectoria A].
	\UCpaso verifica que el teléfono sea numérico y de longitud de 8 o más caracteres. [Trayectoria A].
	\UCpaso verifica que el correo electrónico sea válido. [Trayectoria A].
	\UCpaso verifica que el correo electrónico no pertenezca a un paciente ya registrado. [Trayectoria B].
	\UCpaso verifica que la contraseña tenga longitud de 8 o más caracteres. [Trayectoria A].
	\UCpaso confirma la operación mostrando un mensaje en pantalla
\end{UCtrayectoria}

\begin{UCtrayectoriaA}{A}{Un campo del formulario contiene datos incorrectos}
	\UCpaso notifica el campo incorrecto al actor mostrando el mensaje {\bf MSG3a-} ``Datos incorrectos''.
	\UCpaso muestra la pantalla UI3 Registrar paciente.
	\UCpaso continúa transacción desde el paso 5.
\end{UCtrayectoriaA}
		
\begin{UCtrayectoriaA}{B}{Ya existe un paciente registrado con el correo electrónico ingresado}
	\UCpaso notifica el campo de correo electrónico mostrando el mensaje {\bf MSG3b-}  ``Correo electrónico ya existente en la base de datos''.
	\UCpaso solicita ingresar un correo electrónico diferente en el formulario de UI3 Registrar paciente.
	\UCpaso[\UCactor] ingresa otro correo electrónico en el campo ``Correo electrónico'' del formulario.
	\UCpaso continúa la transacción desde el paso 9.
\end{UCtrayectoriaA}
	%------------ empieza otro caso
	\begin{UseCase}{CU6}{ Consultar citas dadas por médico  
}{Se consulta el número de citas que el médico ha dado en el día y el mes. 
propósito: Conocer el número de citas que el médico ha dado en el día y en el mes para calcular su salario y tener un registro de éstas.}
		\UCitem{Versión}{0.1}
	    \UCitem{Actor}{Gerente}
	    \UCitem{Propósito}{Conocer el número de citas que el médico ha dado en el día y en el mes para calcular su salario y tener un registro de éstas.}
	    \UCitem{Entradas}{-}
	    \UCitem{Origen}{-}
	    \UCitem{Salidas}{Número de citas por día y en el mes dadas por cada médico.
}
	    \UCitem{Destino}{Pantalla del actor}
	   	\UCitem{Precondiciones}{Haber iniciado sesión y haber sido identificado por el sistema como tipo de usuario Gerente
}
	    \UCitem{Postcondiciones}{Ninguna}
	    \UCitem{Observaciones}{Ninguna}
	    \UCitem{Errores}{
	    	\begin{itemize}
				\item ER1
	    	\end{itemize}

	    }
        	    \UCitem{Tipo de ejecución}{Secundaria, viene de CU1 iniciar sesión}

	\end{UseCase}
\begin{UCtrayectoria}{Principal}
	\UCpaso[\UCactor] selecciona Consultar el número de citas haciendo click en el botón ``Consultar citas''.
	\UCpaso obtiene la lista de médicos y el número de citas que ha tenido ese día y ese mes.
	\UCpaso informa la información al usuario desplegándose en una tabla.
\end{UCtrayectoria}

	%------------ empieza otro caso
	\begin{UseCase}{CU7}{Consultar ventas realizadas}{El actor obtiene la lista de las medicinas vendidas en el mes.}
		\UCitem{Versión}{0.1}
	    \UCitem{Actor}{Gerente}
	    \UCitem{Propósito}{Conocer las medicinas que fueron vendidas por los cajeros.}
	    \UCitem{Entradas}{
	    \begin{itemize}
			\item Mes
         \end{itemize}
	    }
	    \UCitem{Origen}{Input tipo select:Mes}
	    \UCitem{Salidas}{Lista de productos vendidos}
	    \UCitem{Destino}{Pantalla}
	   	\UCitem{Precondiciones}{Haber iniciado sesión y haber sido identificado como usuario tipo gerente. 
}
	    \UCitem{Postcondiciones}{Ninguna}
	    \UCitem{Observaciones}{El año el sistema lo determinara automáticamente}
	    \UCitem{Errores}{
	    	\begin{itemize}
				\item ER1
	    	\end{itemize}

	    }
	\end{UseCase}
\begin{UCtrayectoria}{Principal}
	\UCpaso[\UCactor] selecciona visualizar el Reporte de ventas dando click en el botón ``Reporte de ventas''.
	\UCpaso despliega una pantalla con un campo de formulario para seleccionar el mes que el actor desea consultar. [Trayectoria A].
	\UCpaso[\UCactor] selecciona el mes que desea consultar. [Trayectoria A].
	\UCpaso despliega una lista de los productos vendidos con sus respectivas unidades vendidas.
\end{UCtrayectoria}

\begin{UCtrayectoriaA}{A}{No hay ventas registradas}
	\UCpaso muestra el mensaje ``No hay ventas registradas''.
\end{UCtrayectoriaA}   
	%------------ empieza otro caso
	\begin{UseCase}{CU11}{Crear tratamiento}{El actor le asigna un tratamiento a seguir al paciente que está siendo atendido por él.}
		\UCitem{Versión}{0.1}
	    \UCitem{Actor}{Médico}
	    \UCitem{Propósito}{Proporcionar al paciente una descripción de actividades a realizar y una lista de medicamentos para mejorar su malestar.
}
	    \UCitem{Entradas}{
		\begin{itemize}
        \item E-mail : cadena de texto compuesta de 4 partes : 
        \begin{itemize}
            \item cadena de texto que el actor proporciona.
            \item carácter ‘@’
            \item cadena que identifica al servidor que brinda el servicio de correo electrónico
        	\item carácter ‘.’ y dominio
		\end{itemize}
        \item descripción del tratamiento: Cadena extensa de caracteres. 
		\item Nombre completo: Cadena de caracteres compuesto por: Nombre(s) Apellido paterno y apellido Materno
		\item edad:Número entero
    \end{itemize}
	    }
	    \UCitem{Origen}{
        	\begin{itemize}
				\item Teclado: correo electrónico, descripción del tratamiento.
				\item repositorio de datos: Nombre, apellido Paterno y Materno, edad.

			\end{itemize}
        }
	    \UCitem{Salidas}{Mensaje “Se ha vinculado la receta a la cuenta del paciente”}
	    \UCitem{Destino}{
        	\begin{itemize}
				\item Mensaje en pantalla
                \item Tratamiento al repositorio de datos asignado al paciente.
			\end{itemize}
        }
	   	\UCitem{Precondiciones}{El paciente debe estar registrado, el paciente debe tener un expediente médico. 
}
	    \UCitem{Postcondiciones}{ El paciente podrá realizar compra de medicamentos en la farmacia, el paciente podrá consultar un nuevo tratamiento en la lista de sus tratamientos. }
	    \UCitem{Observaciones}{En caso de que el paciente no tenga un expediente el médico podrá crear uno nuevo.
}
	    \UCitem{Errores}{
	    	\begin{itemize}
				\item ER1
	    	\end{itemize}

	    }
        	    \UCitem{Tipo de ejecución}{Secundaria, viene de CU1 iniciar sesión}

\end{UseCase}
\begin{UCtrayectoria}{Principal}
	\UCpaso[\UCactor] selecciona crear una nueva receta dando click en el botón ``Nueva receta''.
	\UCpaso muestra un formulario y solicita los datos corresponientes.
	\UCpaso[\UCactor] proporciona el correo del paciente al sistema. [CU4].
	\UCpaso completa el nombre, apellido Paterno, apellido Materno, edad del paciente y los muestra en pantalla.
	\UCpaso[\UCactor] proporciona el tratamiento al paciente.
	\UCpaso[\UCactor] presiona el botón ``Enviar tratamiento''.
	\UCpaso guarda el tratamiento en el repositorio de datos y muestra el mensaje especificado en las salidas de este caso de uso.
\end{UCtrayectoria}
	%------------ empieza otro caso
	\begin{UseCase}{CU12}{Cancelar Cita
}{El paciente desea cancelar una cita agendada previamente por x razón. 
}
		\UCitem{Versión}{0.1}
	    \UCitem{Actor}{Paciente}
	    \UCitem{Propósito}{ Cancelar una cita agendada y disponibilizar fecha de dicha cita. 
}
	    \UCitem{Entradas}{
	    -
	    }
	    \UCitem{Origen}{-}
	    \UCitem{Salidas}{Mensaje de estatus de operación.
}
	    \UCitem{Destino}{
        \begin{itemize}
			\item Mensaje en pantalla
            \item Operación al repositorio de datos
		\end{itemize}
        }
	   	\UCitem{Precondiciones}{estar registrado en el sistema, haber iniciado sesión, haber agendado una cita, haber seleccionado la opción de “consultar mis citas”}
	    \UCitem{Postcondiciones}{
        \begin{itemize}
			\item El paciente tiene una cita menos.
            \item Nueva fecha disponible.
		\end{itemize}
        }
	    \UCitem{Observaciones}{ Una vez hecha la cancelación no se puede deshacer la operación.
}
	    \UCitem{Errores}{
	    	\begin{itemize}
				\item ER1
	    	\end{itemize}

	    }
     \UCitem{Tipo de ejecución}{Secundaria viene de CU4}

	\end{UseCase}
\begin{UCtrayectoria}{Principal}
	\UCpaso[\UCactor] selecciona cancelar una cita dando click en el botón ``Cancelar cita''.
	\UCpaso pide una confirmación al actor de dicho acto.
	\UCpaso[\UCactor] presiona el botón ``Estoy seguro''. [Trayectoria A].
	\UCpaso informa la cita cancelada.
\end{UCtrayectoria}

\begin{UCtrayectoriaA}{A}{Se presiona el botón ``No estoy seguro''}
	\UCpaso regresa al paso 1 de la trayectoria principal.
\end{UCtrayectoriaA}
	%------------ empieza otro caso
	\begin{UseCase}{CU13}{Registrar pago cita
}{Se almacena el pago de una cita}
		\UCitem{Versión}{0.1}
	    \UCitem{Actor}{Cajero}
	    \UCitem{Propósito}{El paciente puede ingresar a su cita y a su vez poder consultar el pago posteriormente}
	    \UCitem{Entradas}{
	    \begin{itemize}
			\item Fecha y hora de pago: Fecha
	    	\item Monto de pago: Número positivo con dos decimales
			\item ID de consulta: Entero positivo
		\end{itemize}
	    }
	    \UCitem{Origen}{
        \begin{itemize}
			\item Fecha y hora de pago: Se obtienen automáticamente del sistema
            \item Monto de pago: Desde el teclado
            \item ID de consulta: Desde el teclado
		\end{itemize}
        }
	    \UCitem{Salidas}{
        	\begin{itemize}
				\item Mensaje de status de la operación
                \item Operación de tipo actualizar en repositorio de datos
			\end{itemize}
        }
	    \UCitem{Destino}{
        	\begin{itemize}
				\item Mensaje en pantalla
                \item Operación a repositorio de datos
			\end{itemize}
        }
	   	\UCitem{Precondiciones}{ El paciente debe tener una cita agendada
}
	    \UCitem{Postcondiciones}{El sistema registra el pago y la consulta se registra como “pagada”}
	    \UCitem{Observaciones}{}
	    \UCitem{Errores}{
	    	\begin{itemize}
				\item ER1: Si un campo del formulario contiene datos incorrectos se muestra un mensaje para notificar al usuario y después se continúa con el flujo desde el paso 5
                \item ER2: Si un campo requerido está vacío, se muestra un mensaje para notificar al usuario y se continúa el flujo desde el paso 4
				\item ER3: Si la consulta ingresada no existe, se muestra un mensaje para notificar al usuario y se continúa el flujo desde el paso 4
	    	\end{itemize}

	    }
	    \UCitem{Tipo de ejecución}{Secundaria, viene de CU1 iniciar sesión.}
        	    \UCitem{Tipo de ejecución}{Secundaria, viene de CU1 iniciar sesión}

	\end{UseCase}
\begin{UCtrayectoria}{Principal}
	\UCpaso[\UCactor] solicita modificar el estatus de las citas agendadas  haciendo click sobre el botón ``Agregar pago de cita''.
	\UCpaso solicita al actor el monto del pago y el ID de la consulta mostrando un formulario.
	\UCpaso[\UCactor] informa al sistema el monto del pago y el ID de la consulta escribiendolos en el formulario.
	\UCpaso[\UCactor] confirma la operación haciendo click en el botón ``Registrar pago''.
	\UCpaso verifica que los datos requeridos hayan sido proporcionados.
	\UCpaso verifica que los datos proporcionados tengan un formato correcto.
	\UCpaso verifica que el ID de consulta exista.
	\UCpaso almacena el pago guardándolo en la base de datos.
	\UCpaso obtiene la consulta haciendo una solicitud al repositorio de datos.
	\UCpaso cambia el estado de la consulta a “pagada”.
	\UCpaso almacena el estatus de la consulta en el repositorio de datos.
	\UCpaso notifica el resultado de la operación mostrando la información del pago.
\end{UCtrayectoria}    
	%------------ empieza otro caso
	\begin{UseCase}{CU14}{Registrar venta de medicina. 
}{ Se hace la venta de medicamentos.}
		\UCitem{Versión}{0.1}
	    \UCitem{Actor}{Cajero}
	    \UCitem{Propósito}{Tener un registro de los medicamentos vendidos para poder consultar las ventas hechas en la farmacia. 
}
	    \UCitem{Entradas}{
	    \begin{itemize}
			\item Código de barras del medicamento.
		\end{itemize}
	    }
	    \UCitem{Origen}{
        	\begin{itemize}
				\item Lector de código de barras: Código de barras del medicamento 
			
			\end{itemize}
        }
	    \UCitem{Salidas}{
        	\begin{itemize}
				\item  Mensaje de estatus de transacción
           		\item Operación en repositorio de datos
				\end{itemize}
        }
	    \UCitem{Destino}{
        	\begin{itemize}
				\item Mensaje en pantalla
                \item Operación a repositorio de datos
			\end{itemize}
        }
	   	\UCitem{Precondiciones}{Haber iniciado sesión como cajero.}
	    \UCitem{Postcondiciones}{ El sistema restará una unidad a la cantidad disponible de productos.}
	    \UCitem{Observaciones}{Los productos deben estar registrados en el repositorio de datos.}
	    \UCitem{Errores}{
	    	\begin{itemize}
				\item ER1
				\item ERR2: se ha terminado el producto
				\begin{itemize}
					\item Mostrar mensaje : ”Lo siento ese producto no está disponible, favor de registrarlo.”

				\end{itemize}
	    	\end{itemize}

	    }
	    \UCitem{Tipo de ejecución}{Secundaria, viene de CU1 iniciar sesión}
	\end{UseCase}
\begin{UCtrayectoria}{Principal}
	\UCpaso[\UCactor] da click en en botón ``Caja registradora Farmacia''.
	\UCpaso muestra la pantalla donde se harán las ventas de la farmacia.
	\UCpaso espera por un código de barras.
	\UCpaso[\UCactor] introduce el código de barras ya sea por teclado o por lector de código de barras. [ERR2]
	\UCpaso actualiza la lista de productos para la venta.
	\UCpaso[\UCactor] termina de ingresar productos. 
	\UCpaso[\UCactor] da click en el botón ``Realizar venta''.
\end{UCtrayectoria}







