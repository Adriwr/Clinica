%-------------------------------------- COMIENZA descripción del caso de uso.

\begin{UseCase}{CU22}{Modificar Inventario}{El cajero cambiará las cantidades de los medicamentos disponibles para la venta. El sistema modificará las cantidades registradas.}
	\UCitem{Versión}{0.1}
    \UCitem{Actor}{Cajero}
    \UCitem{Propósito}{Actualizar las cantidades disponibles registradas en el sistema cuando se compren medicamentos en la farmacia.}
    \UCitem{Entradas}{
        \begin{itemize}
            \item ID del producto: Cadena de dígitos. Requerido.
            \item Cantidad: Número entero no negativo. Requerido.
        \end{itemize} 
    }
    \UCitem{Origen}{
        \begin{itemize}
            \item ID del producto: Petición HTTP.
	        \item Cantidad: Teclado.
        \end{itemize}
    }
    \UCitem{Salidas}{
        \begin{itemize}
	        \item Mensaje MSG22a ``Cantidad registrada correctamente''.
		\end{itemize}
	}
    \UCitem{Destino}{
		\begin{itemize}
			\item IU23 Consultar Inventario.
		\end{itemize}
	}
	
   	\UCitem{Precondiciones}{
   	    \begin{itemize}
			\item Existen medicamentos registrados en el sistema.
   	    \end{itemize}
   	}
    \UCitem{Postcondiciones}{Ninguna}
    \UCitem{Observaciones}{Ninguna}
    \UCitem{Errores}{
    	\begin{itemize}
    	\item ERR1 - Producto no encontrado:
    		\begin{itemize}
    			\item El sistema mestra el mensaje MSG22d ``Producto no encontrado''.
    			\item Regresa al punto 2.
    		\end{itemize}
    	\item ERR2 - Cantidad ingresada no válida:
	    	\begin{itemize}
				\item El sistema muestra el mensaje MSG22b ``Ingrese una cantidad válida''.
				\item Regresa a punto 4.
			\end{itemize}
		\item ERR3 - Cantidad no ingresada:
			\begin{itemize}
				\item El sistema muestra el mensaje MSG22c ``Ingrese la cantidad''.
				\item Regresa al punto 4.
			\end{itemize}
		\end{itemize}
	}
	\UCitem{Tipo de ejecución}{Secundaria, viene de CU23 Consultar Inventario}
	\UCitem{Prioridad}{Media}
	\UCitem{Volatilidad}{Media}
	\UCitem{Madurez}{Alta}
	\UCitem{Estado}{En revisión}
	\UCitem{Autor}{Demis Gómez Moncada}
	\UCitem{Revisor}{}
	
\end{UseCase}

\begin{UCtrayectoria}{}
		\UCpaso[\UCactor] selecciona el medicamento que va a modificar dando click en el botón ``Modificar''.
		\UCpaso solicita los datos del medicamento al repositorio de datos. [ERR1].
		\UCpaso muestra la pantalla IU22 Modificar inventario.
		\UCpaso solicita la nueva cantidad del medicamento.
		\UCpaso [\UCactor] ingresa la nueva cantidad en el campo `cantidad' del formulario.[ERR2] [ERR3].
		\UCpaso actualiza la cantidad en inventario del medicamento.
		\UCpaso registra la nueva cantidad en el repositorio de datos.
\end{UCtrayectoria}
%-------------------------------------- TERMINA descripción del caso de uso.