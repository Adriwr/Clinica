% \IUref{IUAdmPS}{Administrar Planta de Selección}
% \IUref{IUModPS}{Modificar Planta de Selección}
% \IUref{IUEliPS}{Eliminar Planta de Selección}

% 


% Copie este bloque por cada caso de uso:
%-------------------------------------- COMIENZA descripción del caso de uso.

%\begin{UseCase}[archivo de imágen]{UCX}{Nombre del Caso de uso}{
\begin{UseCase}{Cu7}{Consultar Ventas}{El actor desea consultar las ventas realizadas en la farmacia.}
    \UCitem{Versión}{0.2}
    \UCitem{Actor}{Gerente}
    \UCitem{Propósito}{Conocer el número de productos vendidos en el mes o en el día.}
    \UCitem{Entradas}{
        \begin{itemize}
            \item Fecha: Formato dd/mm/aaaa 
        \end{itemize} 
    }
    \UCitem{Origen}{
        \begin{itemize}
            \item Desde el sistema.
        \end{itemize}
    }
    \UCitem{Salidas}{
        \begin{itemize}
            \item Lista de productos vendidos
        \begin{itemize}
            \item *Nombre: Constituido por nombre del producto
            \item Precio del producto: tipo de dato con decimales.
            \item Número de productos vendidos: Número entero
           
        \end{itemize}
        \end{itemize}
        
           }

    \UCitem{Destino}{Pantalla}
    \UCitem{Precondiciones}{
        \begin{itemize}
            \item Haber sido identificado por el sistema como tipo de actor gerente. 
        \end{itemize}
    }
    \UCitem{Postcondiciones}{Ninguna}
    \UCitem{Observaciones}{
    	\begin{itemize}
    		\item Si no ha habido ventas del producto el sistema debe poner un '0'.
    		\item El sistema requerirá partes de la fecha dependiendo si es por día o por mes; por mes ocupará el mes y el año. Por día ocupará el día,mes,año.
    	\end{itemize}
 
    }
    \UCitem{Errores}{
    Ninguno
    }
    \UCitem{Tipo de ejecución}{Secundaria, viene de CU1 iniciar sesión}
    \UCitem{Prioridad}{Mediana}
    \UCitem{Volatilidad}{Baja}
    \UCitem{Madurez}{Baja}
    \UCitem{Estado}{En edición}
    \UCitem{Autor}{Rubén Murga Dionicio}
    \UCitem{Revisor}{David Pacheco Soto}
    
\end{UseCase}

\begin{UCtrayectoria}{Gerente  [ventas diarias]}
        \UCpaso[\UCactor] selecciona "Ventas de hoy" dando click en dicho apartado.
        
        \UCpaso Muestra una lista de los productos vendidos en el día con los datos de salida especificados en dicho apartado [TA1]

\end{UCtrayectoria}

\begin{UCtrayectoria}{\begin{UCtrayectoria}{Gerente  [ventas diarias]}
        \UCpaso[\UCactor] selecciona "Ventas de hoy" dando click en dicho apartado.
        
        \UCpaso informa al actor un listado  de los productos vendidos en el día mostrando la lista con los datos de salida especificados en dicho apartado [TA1]


\end{UCtrayectoria}Gerente  [ventas mensuales]}
        \UCpaso[\UCactor] selecciona "Ventas mensuales" dando click en dicho apartado.
        
        \UCpaso informa al actor un listado  de los productos vendidos en el mes mostrando la lista con los datos de salida especificados en dicho apartado [TA1]

\end{UCtrayectoria}

\begin{UCtrayectoriaA}{TA1}{Producto sin ventas}
    \UCpaso El sistema muestra los datos que no tienen ventas con valor de '0' en el campo de número de ventas
    
%-------------------------------------- TERMINA descripción del caso de uso.
