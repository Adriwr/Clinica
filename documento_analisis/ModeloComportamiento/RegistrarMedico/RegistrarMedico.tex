% \IUref{IUAdmPS}{Administrar Planta de Selección}
% \IUref{IUModPS}{Modificar Planta de Selección}
% \IUref{IUEliPS}{Eliminar Planta de Selección}

% 


% Copie este bloque por cada caso de uso:
%-------------------------------------- COMIENZA descripción del caso de uso.

%\begin{UseCase}[archivo de imágen]{UCX}{Nombre del Caso de uso}{
\begin{UseCase}{Cu19}{Registrar médico}{El actor desea registrar un nuevo médico.}
	\UCitem{Versión}{0.1}
    \UCitem{Actor}{Gerente}
    \UCitem{Propósito}{Que el médico pueda atender pacientes y que pueda utilizar las funcionalidades del sistema para médico.}
    \UCitem{Entradas}{
    	\begin{itemize}
    		\item Email del médico: cadena de texto compuesta de 4 partes [requerido]:  
    		\begin{itemize}
    			\item cadena de texto.
    			\item carácter ‘@’
    			\item cadena que identifica al servidor que brinda el servicio de correo electrónico
    			\item carácter ‘.’ y dominio
    		\end{itemize}
    		\item *Contraseña:  Cadena de longitud mínima de 8 caracteres
			\item Lista de datos personales [todos requeridos]
        \begin{itemize}
            \item Nombre: Constituido por nombre(s) y apellidos.Cádena de caracteres
            \item Sexo  Caracter 'H' o 'M' (hombre,mujer)
            \item Fecha de nacimiento: Formato de fecha DD/MM/AAAA
            \item Teléfono: Cadena de 8 a 20 dígitos
            \item Teléfono de emergencia: Cadena de 8 a 20 dígitos
		
		\end{itemize}
		\end{itemize}
    }
    \UCitem{Origen}{
    	\begin{itemize}
    		\item Desde teclado: todos los campos.
    	\end{itemize}
    }
    \UCitem{Salidas}{
    	\begin{itemize}
    		\item Mensaje de confirmación ``Médico registrado correctamente''
    	\end{itemize}
    }
    
    \UCitem{Destino}{Pantalla}
   	\UCitem{Precondiciones}{
   	    \begin{itemize}
	   	    \item Que no exista un médico registrado con el mismo email.
   	    \end{itemize}
   	}
    \UCitem{Postcondiciones}{Se guardan los datos del médico en el sistema}
    \UCitem{Observaciones}{Ninguna}
    \UCitem{Errores}{
    	\begin{itemize}
    		\item ERR1: Si no se completaron todos los datos requeridos, se informa al actor mostrando el mensaje ``Complete todos los campos'' y se continúa desde el paso 2
    		\item ERR2: Si los datos requeridos no tienen el formato correcto, se informa al actor mostrando el mensaje ``Formato de campos inválido'' y se continúa desde el paso 2 
    	\end{itemize}
    }
	\UCitem{Tipo de ejecución}{Primaria}
	\UCitem{Prioridad}{Media}
	\UCitem{Volatilidad}{Baja}
	\UCitem{Madurez}{Baja}
	\UCitem{Estado}{En edición}
	\UCitem{Autor}{David Alberto Pacheco Soto}
	\UCitem{Revisor}{}
	
\end{UseCase}

\begin{UCtrayectoria}{Gerente}
		\UCpaso[\UCactor] solicita registrar un médico dando click en \IUbutton{Registrar médico}.
		\UCpaso Solicita los datos listados en entradas mostrando la IU
		\UCpaso [\UCactor] Proporciona los datos
		\UCpaso [\UCactor] Confirma los datos haciendo click en el botón \IUbutton{Finalizar registro de médico}
		\UCpaso Verifica que todos los datos requeridos hayan sido completados
		\UCpaso Verifica que los datos proporcionados tengan el formato requerido
		\UCpaso Verifica que el correo electrónico ingresado no esté registrado [TA1]
		\UCpaso Almacena los datos del médico 
		\UCpaso Informa el resultado de la transacción
\end{UCtrayectoria}

\begin{UCtrayectoriaA}{TA1}{Ya existe un médico registrado con ese correo electrónico}
	\UCpaso Muestra mensaje de error ``El correo electrónico proporcionado ya está registrado''
	\UCpaso Continúa la ejecución desde paso 2
	
\end{UCtrayectoriaA}

%-------------------------------------- TERMINA descripción del caso de uso.
