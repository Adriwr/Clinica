%\begin{UseCase}[archivo de imágen]{UCX}{Nombre del Caso de uso}{
	\begin{UseCase}{Cu10}{Realizar pago de farmacia}{El actor registra el pago de un medicamento en la farmacia.}
	\UCitem{Versión}{0.1}
    \UCitem{Actor}{Cajero}
    \UCitem{Propósito}{Que el medicamento que se sale del inventario sea actualizado tanto en la cantidad del inventario como el ingreso de éste para la clínica.}
    \UCitem{Entradas}{
        \begin{itemize}        	
        	\item Nombre de medicamento: Cadena de texto con el nombre del medicamento. [Requerido]        	
        \end{itemize} 
    }
    \UCitem{Origen}{
        \begin{itemize}
            \item Desde teclado: Nombre del medicamento            
        \end{itemize}
    }
    \UCitem{Salidas}{
        \begin{itemize}
	        \item Comprobante de compra: Archivo PDF que contiene:
	        \begin{itemize}
	        	\item Información de compra:
	        	\begin{itemize}
	        		\item Fecha de compra: Fecha en formato dd/mm/yyyy
	        		\item Hora de cmpra: Hora con formato hh:mm	        		
	        	\end{itemize}
	        	\item Información de compra:
            \begin{itemize}
              \item Nombre de los productos adquiridos.
              \item Precio unitario de cada producto.
              \item Cantidad adquirida.           
            \end{itemize}
	        \end{itemize}
	        \item Mensaje de confirmación: Cadena ''Pago registrado correctamente''
	        
	    \end{itemize}
    }
    \UCitem{Destino}{
	    \begin{itemize}
	    	\item Pantalla: Mensaje de confirmación y comprobante de compra	    	
	    \end{itemize}
    }
   	\UCitem{Precondiciones}{
        \begin{itemize}            
            \item Que el actor tenga iniciada su sesión            
        \end{itemize}
   	}
    \UCitem{Postcondiciones}{
	    	\begin{itemize}
	    		\item El producto se actualiza en el repositorio de datos tanto en su cantidad como en el valor de ingreso de dicho medicamento para la clínica. 
	    	\end{itemize}
    }
    \UCitem{Observaciones}{El actor podrá realizar la búsqueda de la cantidad de medicamentos que se necesiten.}
    \UCitem{Errores}{
    	\begin{itemize}
    		\item ERR1: Si no se completaron todos los datos requeridos, se informa al actor mostrando el mensaje ``Complete todos los campos'' y se continúa desde el paso 2
    		\item ERR2: Si los datos requeridos no tienen el formato correcto, se informa al actor mostrando el mensaje ``Formato de campos inválido'' y se continúa desde el paso 2 
    	\end{itemize}
	}
	\UCitem{Tipo de ejecución}{Primaria}
	\UCitem{Prioridad}{Alta}
	\UCitem{Volatilidad}{Baja}
	\UCitem{Madurez}{Baja}
	\UCitem{Estado}{Terminado}
	\UCitem{Autor}{David Alberto Pacheco Soto}
	\UCitem{Revisor}{Saúl Uriel Trujillo García}
	
\end{UseCase} 

\begin{UCtrayectoria}{Cajero}
		\UCpaso[\UCactor] Solicita realizar pago de cita haciendo click sobre el botón \IUbutton{Realizar pago de cita}
		\UCpaso Solicita el ID de la cita mostrando \IUref{1}{Realizar pago de cita}
		\UCpaso[\UCactor] Proporciona el ID de la cita ingresándolo en el campo de texto correspondiente.[RN13][RN8]
		\UCpaso[\UCactor] Confirma operación haciendo click sobre el botón \IUbutton{Registrar pago}
		\UCpaso Verifica que se hayan completado datos requeridos
		\UCpaso Verifica que los datos completados tengan el formato correcto
		\UCpaso Obtiene la cita mediante el ID de cita ingresado [TA1]
		\UCpaso Verifica que la fecha y la hora de la cita no sean posteriores a la hora y fecha actuales [TA2]
		\UCpaso Verifica que la cita no tenga el estado de ``pagada'' [TA3]
		\UCpaso Modifica el estado de la cita a ``pagada''
		\UCpaso Obtiene al paciente que agendó la cita 
		\UCpaso Confirma operación mostrando mensaje de confirmación, datos de cita y enviándo archivo PDF del comprobante  al correo electrónico del paciente
\end{UCtrayectoria}

\begin{UCtrayectoriaA}{TA1}{Cita no existe}
	\UCpaso Informa al acto mostrando el mensaje ``Cita no existe''
	\UCpaso Continúa trayectoria desde paso 2
	
\end{UCtrayectoriaA}
	
\begin{UCtrayectoriaA}{TA2}{La fecha o la hora son posteriores a la fecha y hora actual}
	\UCpaso Informa al acto mostrando el mensaje ``La cita expiró''
	\UCpaso Continúa trayectoria desde paso 2
	
\end{UCtrayectoriaA}

\begin{UCtrayectoriaA}{TA3}{Cita tiene estado de pagada}
	\UCpaso Informa al acto mostrando el mensaje ``Cita ya está pagada''
	\UCpaso Continúa trayectoria desde paso 2
	
\end{UCtrayectoriaA}
%-------------------------------------- TERMINA descripción del caso de uso.