%\begin{UseCase}[archivo de imágen]{UCX}{Nombre del Caso de uso}{
	\begin{UseCase}{Cu10}{Realizar pago de farmacia}{El actor registra el pago de un medicamento en la farmacia.}
	\UCitem{Versión}{0.1}
    \UCitem{Actor}{Cajero}
    \UCitem{Propósito}{Que el medicamento que se sale del inventario sea actualizado tanto en la cantidad del inventario como el ingreso que representa la salida de éste para la clínica, además de generar un comprobante de compra para el paciente que adquirió la(s) medicina(s).}
    \UCitem{Entradas}{
        \begin{itemize}        	
        	\item Nombre de medicamento: Cadena de texto con el nombre del medicamento que será utilizado para la búsqueda en el repositorio de datos. [Requerido]. 
           \item Cantidad de producto a registrar: Número entero positivo mayor a 0.[Requerido].
        \end{itemize} 
    }
    \UCitem{Origen}{
        \begin{itemize}
            \item Desde teclado: Nombre del medicamento.
            \item Seleccinable: Cantidad del producto a registrar.
        \end{itemize}
    }
    \UCitem{Salidas}{
        \begin{itemize}
	        \item Comprobante de compra: Archivo PDF que contiene:
	        \begin{itemize}
	        	\item Información de compra:
	        	\begin{itemize}
	        		\item Fecha de compra: Fecha en formato dd/mm/yyyy
	        		\item Hora de cmpra: Hora con formato hh:mm	        		
	        	\end{itemize}
	        	\item Información de compra:
            \begin{itemize}
              \item Nombre de los productos adquiridos.
              \item Precio unitario de cada producto.
              \item Cantidad adquirida.           
            \end{itemize}
	        \end{itemize}
	        \item Mensaje de confirmación: Cadena ''Pago registrado correctamente''
	        
	    \end{itemize}
    }
    \UCitem{Destino}{
	    \begin{itemize}
	    	\item Pantalla: Mensaje de confirmación y comprobante de compra	    	
	    \end{itemize}
    }
   	\UCitem{Precondiciones}{
        \begin{itemize}            
            \item Que el actor tenga iniciada su sesión            
        \end{itemize}
   	}
    \UCitem{Postcondiciones}{
	    	\begin{itemize}
	    		\item Los datos que se tienen sobre el o los productos se actualiza en el repositorio de datos tanto en su cantidad como en el valor de ingreso de dicho medicamento para la clínica. 
	    	\end{itemize}
    }
    \UCitem{Observaciones}{El actor podrá realizar la búsqueda de la cantidad de medicamentos que se necesiten.}
    \UCitem{Errores}{
    	\begin{itemize}
    		\item ERR1: Si no se completaron todos los datos requeridos, se informa al actor mostrando el mensaje ''Ingrese todos los datos requeridos para la compra'' y se continúa desde el paso 2
    		\item ERR2: Si el medicamento que se desea buscar no se encuentra en los registros del repositorio de datos, se informa al actor mostrando el mensaje ''No se encuentraron coincidencias...'' y se continúa desde el paso 2 
    	\end{itemize}
	}
	\UCitem{Tipo de ejecución}{Primaria}
	\UCitem{Prioridad}{Alta}
	\UCitem{Volatilidad}{Baja}
	\UCitem{Madurez}{Baja}
	\UCitem{Estado}{En edición}
	\UCitem{Autor}{Saúl Uriel Trujillo García}
	\UCitem{Revisor}{Demis Gómez Moncada}
	
\end{UseCase} 

\begin{UCtrayectoria}{Cajero}
		\UCpaso[\UCactor] Solicita realizar pago de medicamento haciendo click sobre el botón \IUbutton{Realizar pago de medicamento} de la \IUref{Dashboard}
    \UCpaso Verifica si existe alguna compra pendiente en la sesión realizada por el actor anteriormente. [TA1]
    
    \UCpaso Muestra el formulario para la búsqueda del medicamento que se encuentra en la \IUref{1}{Formulario de busqueda de medicamento}
		\UCpaso[\UCactor] Ingresa el dato de entrada ''Nombre del medicamento''.
    \UCpaso[\UCactor] Enviar los datos requeridos para la búsqueda de medicamento haciendo click sobre el botón \IUbutton{Buscar} de la \IUref{1}{Formulario de búsqeuda de medicamento}
		\UCpaso Obtiene los datos enviados por el actor y verifica que el dato 
		\UCpaso[\UCactor] Confirma operación haciendo click sobre el botón \IUbutton{Registrar pago}
		\UCpaso Verifica que se hayan completado datos requeridos
		\UCpaso Verifica que los datos completados tengan el formato correcto
		\UCpaso Obtiene la cita mediante el ID de cita ingresado [TA1]
		\UCpaso Verifica que la fecha y la hora de la cita no sean posteriores a la hora y fecha actuales [TA2]
		\UCpaso Verifica que la cita no tenga el estado de ``pagada'' [TA3]
		\UCpaso Modifica el estado de la cita a ``pagada''
		\UCpaso Obtiene al paciente que agendó la cita 
		\UCpaso Confirma operación mostrando mensaje de confirmación, datos de cita y enviándo archivo PDF del comprobante  al correo electrónico del paciente
\end{UCtrayectoria}

\begin{UCtrayectoriaA}{TA1}{Existen medicamentos pendientes de compra}
  \UCpaso Obtiene los medicamentos que se tienen pendientes en la sesión.
%--  \UCpaso Muestra sobre la \UIref 
	\UCpaso Informa al acto mostrando el mensaje ``Cita no existe''
	\UCpaso Continúa trayectoria desde paso 2
	
\end{UCtrayectoriaA}
	
\begin{UCtrayectoriaA}{TA2}{La fecha o la hora son posteriores a la fecha y hora actual}
	\UCpaso Informa al acto mostrando el mensaje ``La cita expiró''
	\UCpaso Continúa trayectoria desde paso 2
	
\end{UCtrayectoriaA}

\begin{UCtrayectoriaA}{TA3}{Cita tiene estado de pagada}
	\UCpaso Informa al acto mostrando el mensaje ``Cita ya está pagada''
	\UCpaso Continúa trayectoria desde paso 2
	
\end{UCtrayectoriaA}
%-------------------------------------- TERMINA descripción del caso de uso.