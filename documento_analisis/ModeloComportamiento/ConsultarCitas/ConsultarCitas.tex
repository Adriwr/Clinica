%-------------------------------------- COMIENZA descripción del caso de uso.

\begin{UseCase}{CU3}{Consultar Citas}{El paciente consultará el historial de sus próximas citas agendadas cuando necesite observar la fecha, hora, consultorio y médico de las citas que ha agendado. El sistema mostrará la información de las próximas citas del paciente.}
	\UCitem{Versión}{0.4}
    \UCitem{Actor}{Paciente}
    \UCitem{Propósito}{El paciente pueda observar la fecha, hora, consultorio y médico de sus próximas citas.}
    \UCitem{Entradas}{
        \begin{itemize}
            \item Id del paciente: cadena de caracteres alfanuméricos.
        \end{itemize} 
    }
    \UCitem{Origen}{
        \begin{itemize}
            \item Id del paciente: Variable de sesión de la aplicación.
        \end{itemize}
    }
    \UCitem{Salidas}{
        \begin{itemize}
        \item Lista de citas: Arreglo de estructuras de datos que contienen lo siguiente:
	        \begin{itemize}
				\item Fecha y hora: cadena de caracteres con el formato ``DD/MM/AAAA HH:ii'', donde DD, MM y AAAA son número enteros que representan el día, mes, año , hora y los minutos de la cita, respectivamente.
				\item Consultorio: Número entero que representa el número de consultorio.
				\item Doctor: Cadena de texto compuesta de dos partes, separadas por un espacio:
				\begin{itemize}
					\item Nombre(s) del médico.
					\item Apellidos del médico.
		        \end{itemize}
		        \item Identificador de la cita: oculto.
		    \end{itemize}
		\end{itemize}
	}
    \UCitem{Destino}{
		\begin{itemize}
			\item Fecha y Hora, Consultorio, Doctor: Pantalla UI3 Consultar Citas del paciente.
			\item Identificador de la consulta: Oculta del lado de cliente.
		\end{itemize}
	}
	
   	\UCitem{Precondiciones}{
   	    \begin{itemize}
			\item El paciente ha agendado al menos una cita en el sistema.
   	    \end{itemize}
   	}
    \UCitem{Postcondiciones}{Ninguna}
    \UCitem{Observaciones}{Ninguna}
    \UCitem{Errores}{
    	\begin{itemize}
    	\item ERR1 - No hay citas agendadas:
	    	\begin{itemize}
				\item muestra el MSG3a ``No hay citas agendadas''.
				\item Fin del CU3.
			\end{itemize}
		\end{itemize}
	}
	\UCitem{Tipo de ejecución}{Secundaria, viene de CU1 iniciar sesión}
	\UCitem{Prioridad}{Media}
	\UCitem{Volatilidad}{Baja}
	\UCitem{Madurez}{Media Alta}
	\UCitem{Estado}{Terminado}
	\UCitem{Autor}{Demis Gómez Moncada}
	\UCitem{Revisor}{Adrián Eduardo Galindo García}
	
\end{UseCase}

\begin{UCtrayectoria}{}
		\UCpaso[\UCactor] selecciona mostrar las citas dando click en el botón ``Mostrar citas''.
		\UCpaso solicita al repositorio de datos los datos de las próximas citas del paciente. [ERR1].
		\UCpaso muestra la pantalla UI3 Consultar citas del paciente.
\end{UCtrayectoria}
%-------------------------------------- TERMINA descripción del caso de uso.