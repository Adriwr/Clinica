Introduction al capítulo

%--------------------------------------------------
\section{Glosario de términos}
\begin{description}
\item [Paciente:] Es una persona a la que se le proporciona atención médica. 
\item [Clínica:] Es el lugar que se dedica a la observación y tratamiento médico de pacientes, así como a la venta de medicamentos.
\item [Consultorio:] Habitación ambientada para la revisión de pacientes.
\item [Consulta:] Es el proceso en el que un paciente asiste a un consultorio a ser revisado por un médico. 
\item [Cita:] Es una consulta que está agendada previamente en un horario y consultorio definido.
\item [Expediente (clínico):] Documento que describe la situación y antecedentes médicos de un paciente.
\item [Médico:] Persona encargada de proporcionar atención médica profesional a persona con malestares físicos.
\item [Enfermera:] Persona encargada de proporcionar ayuda a personas nuevas en la clínica. 
\item [Horario Médico:] Horario laboral que se le le fue asignado al médico.
\item [Tratamiento:] Lista de recomendaciones hechas por un médico hacia algún paciente para mejorar su situación de salud.
\item [Farmacia:] Habitación en donde se encuentran los medicamentos y es administrada por los cajeros.
\item [Cajeros:] Persona encargada de hacer el cobro de medicamentos y cobro de cita.
\item [Gerente:] Persona que actúa el administrador del sistema, tiene el poder de modificar los horarios de los médicos y consultar las ventas y cobros hechos en la farmacia.
\end{description}

