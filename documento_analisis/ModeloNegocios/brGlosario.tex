%--------------------------------------------------
\section{Glosario de términos}
\begin{description}
\item [Médico:] Persona encargada de brindar atención médica profesional y especializada a una persona que requiera atención medica.
\item [Enfermera:] Persona encargada de auxiliar a los médicos en diferentes tareas relacionadas a los pacientes. 
\item [Paciente:] Es una persona a la que se le proporciona atención médica. 
\item [Cita:] Es una consulta que está programada en un horario definido, con un medico determinado.
\item [Clínica:] Es el lugar que se dedica al tratamiento médico; además se puede dedicar a la venta de medicamentos.
\item [Consultorio:] Habitación para la revisión de pacientes acondicionada de acuerdo a las necesidades específicas de cada paciente.
\item [Cajeros:] Persona encargada de realizar el cobro de medicamentos así cobro correspondiente a las cita.
\item [Gerente:] Persona encargada de administrar el sistema, tiene el poder de modificar los horarios de los médicos,consultar las ventas, consultar cobros hechos en la farmacia así como poder visualizar los medicamentos en existencia dentro de la farmacia..
\item [Consulta:] Es el proceso en el que un paciente es revisado por un médico especializado. 
\item [Expediente clínico:] Documento que contiene la situación y antecedentes médicos de un paciente.
\item [Horario Médico:] Horario laboral en el que la clínica puede atender a un paciente.
\item [Tratamiento:] Lista de pasos a seguir, realizada por el medico con el fin de mejorar el estado de salud de los pacientes.
\item [Farmacia:] Lugar donde se encuentran almacenados los medicamentos para su venta a los pacientes con receta.
\end{description}

